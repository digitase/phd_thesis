%
% TODOs
%

%
% Writing strategy, top-down
%
% Headings
% Subheadings
% Topic sentences
% End linker sentences

%
% Basic abstract format (Nature)
%
% One or two sentences providing a basic introduction to the field, comprehensible to a scientist in any discipline.
%
% Two to three sentences of more detailed background, comprehensible to scientists in related disciplines.
%
% One sentence clearly stating the general problem being addressed by this particular study.
% i.e. 'the open question' / 'gap the paper fills'
%
% One sentence summarizing the main result (with the words "here we show" or their equivalent).
%
% Two or three sentences explaining what the main result reveals in direct
% comparison to what was thought to be the case previously, or how the main
% result adds to previous knowledge.
%
% One or two sentences to put the results into a more general context.
% i.e. 'broader significance'

%
% Mensh, B., & Kording, K. (2017). Ten simple rules for structuring papers. PLOS Computational Biology, 13(9), e1005619.
% https://doi.org/10.1371/journal.pcbi.1005619
%
%
% The context-content-conclusion (C-C-C) scheme
%
% The beginning sets up the context for the story, while the body (content)
% advances the story towards an ending in which the problems find their
% conclusions. This structure reduces the chance that the reader will wonder "Why
% was I told that?" (if the context is missing) or "So what?" (if the conclusion
% is missing).
%
% Title
% The one central and memorable contribution of the paper.
%
% Introduction
%
% Big problem in science: field domain (context), what field knows (content), remaining gap (conclusion)
% Narrower problem in the subfield
% Narrower gap the paper addresses
%     The introduction should not contain a broad literature review beyond the
%     motivation of the paper. [...] [readers] only need to assess the importance
%     of the claimed gap.
% Summary: our approach, our results
%     The last paragraph of the introduction is special: it compactly summarizes
%     the results, which fill the gap you just established.
%
% Results
%
% First results paragraph:
% Summarizes the overall approach to the problem outlined in the introduction,
% along with any key innovative methods.
%
% Subsequent results paragraphs:
% 1-2 sentences that set up purpose of the paragraph.
% Data and logic.
% Concluding sentence.
%
% Figure captions:
% The title of the figure should communicate the conclusion of the analysis,
% and the legend should explain how it was done.
%
% Discussion
%
% What we found
% How we filled the gap
%     The first discussion paragraph is special in that it generally summarizes the important
%     findings from the results section. Some readers skip over substantial parts of the results, so
%     this paragraph at least gives them the gist of that section
%
% Limitations in filling the gap
% Limitations in generalising
% How these limitations may be addressed
%     Each of the following paragraphs in the discussion section starts by describing an area of
%     weakness or strength of the paper. It then evaluates the strength or weakness by linking it to
%     the relevant literature.
%
% Strengths
% How the paper advances the field by providing new opportunities
% Future areas of research
%     Description of how the paper moves the field forward.

\todo{spell-check}
\todo{make sure package versions are in, and package names are monospace}
\todo{add automatic rounding to x decimal places using num and sisetup}
\todo{collaboration note in italics at start of each chapter}

\listoftodos

