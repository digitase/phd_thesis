%
% Chapter 6
% Discussion
%

\chapter{Discussion}

Tie ch 2 to 3 using baseline predictors?

Limitations, and the perfect study.

A response eqtl is not always a response eqtl

Era of single cell.
    1st
    Single-cell RNA sequencing identifies celltype-specific cis-eQTLs and co-expression QTLs
    https://www.nature.com/articles/s41588-018-0089-9

    "Single-cell eQTLGen Consortium: a personalized understanding of disease"
    https://arxiv.org/abs/1909.12550

    Optimal design of single-cell RNA sequencing experiments for cell-type-specific eQTL analysis
    https://www.biorxiv.org/content/biorxiv/early/2019/09/12/766972.full.pdf

    Single-cell genomic approaches for developing the next generation of immunotherapies Ido Yofe, Rony Dahan and Ido Amit

    % Lähnemann, D., Köster, J., Szczurek, E., McCarthy, D. J., Hicks, S. C.,
    % Robinson, M. D., Vallejos, C. A., Campbell, K. R., Beerenwinkel, N., Mahfouz,
    % A., Pinello, L., Skums, P., Stamatakis, A., Attolini, C. S.-O., Aparicio, S.,
    % Baaijens, J., Balvert, M., Barbanson, B. de, Cappuccio, A., … Schönhuth, A.
    % (2020). Eleven grand challenges in single-cell data science. Genome Biology,
    % 21(1). https://doi.org/10.1186/s13059-020-1926-6

    reQTL detection: bulk, sorted, sc
    current sc will only detect highly expressed genes

Cost-effectiveness and clinical implementation

    if you can identify NRs, what are you going to do about it?

Deep phenotyping
    
    disease specific biobanks e.g. ibd bioresource/predicct

unification
    immunology and vaccine dev: deep phenotyping, small cohorts achieved -> larger cohorts
    human genetics and gwas: large cohorts achieved -> deeper phenotyping

