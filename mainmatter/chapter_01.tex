%
% Chapter 1
%
% Unifies results chapters under the framework of human genetics.

\chapter{Introduction}

Why study human genetics?

Human variation
Nature vs nurture
Causal anchors.
Leveraging natural G variation.

\section{A brief history of complex trait genetics}

Prior to GWAS
Mendelian genetics, family and linkage studies

Twin studies and heritability estimates.

Candidate gene studies (Border et al., 2019)

\subsection{The era of \glsfmtshort{GWAS}}

Common disease, common variant
X years of GWAS

Missing heritability
Rare variation, burden tests

Fine-mapping

Pathway analysis

TWAS
PheWAS\autocite{verma2017CurrentScopeChallenges}
MR

expanding into global populations
polygenic scores

\subsection{Post-\glsfmtshort{GWAS}: interpretation of genetic associations}

Why?
Understand mech.
Drug target prioritisation for disease traits
e.g. of successful GWAS -> drug target

Locus to gene problem, nc variation.
Genome-wide association studies have successfully identified genetic variants associated with immune-mediated disease, the majority of which are non-coding[10 Years of GWAS Discovery].

Under the assumption that the mechanism by which non-coding associations affect disease risk is through their effect on gene expression, a successful way to link associations to their target gene is by statistical colocalisation with eQTL datasets, to determine if the GWAS and eQTL signal share the same causal variant[Co-localization of Conditional eQTL and GWAS Signatures in Schizophrenia].

molecular \glspl{QTL} in context % Many types (review)
Coloc

Context is key

\section{Immunity is a complex trait}
Immune-mediated diseases
Heritability of immune parameters and immune-mediated diseases

\subsection{Genetic factors affecting the healthy immune system}

Why study health?
Factors affecting the healthy immune system.

\subsection{Genetic factors affecting immune response to challenge}

Given the genetic control of the healthy immune system, one can hypothesise that immune response to challenge may also be influenced by genetic factors.

The need for controlled immune challenge in trials.
Studies of natural infection are complicated.
Drug trials as an opportunity: Vaccines and biologics for controlled immune challenge.

\section{Immune response to vaccination}

Vaccination has enormous impact on global health [10.1098/rstb.2013.0433].

Vaccines stimulate the immune system with pathogen-derived antigens to induce effector responses (primarily antigen-specific antibodies) and immunological memory against the pathogen itself.
These effector responses are then be rapidly reactivated in cases of future exposure to the pathogen, mediating long-term protection.

\subsection{Systems vaccinology: from empirical to rational vaccinology}

History of vaccine dev
[summary of low-throughput immunology e.g. animal models]  

However, a vaccine that is highly efficacious in one human population may have significantly lower efficacy in other populations.
[1 statistic on vaccine efficiacy differences e.g. rotavirus]
Particularly challenging populations for vaccination include the young and elderly, immunosuppressed patients, ethnically-diverse populations, and developing countries.
For the majority of licensed vaccines, there is a lack of understanding regarding the molecular mechanisms that underpin this variation in host immune response.
Immunological mechanisms that underpin a specific vaccine's success or failure in a given individual are often poorly understood[Immunological mechanisms of vaccination]. 

rational vacc, where the key is sys vacc
Review of systems vaccinology (pull out of self\_viva\_copypasta)
These systems vaccinology studies often consider longitudinal measurements of the transcriptomic, cellular, cytokine, and antibody immune responses following vaccination[Vaccinology in the era of high-throughput biology.].
Systems vaccinology is the application of -omics technologies to provide a systems-level characterisation of the human immune system after vaccine-perturbation.
Measurements are taken at multiple molecular levels (e.g. genome, transcriptome, proteome), and molecular signatures that correlate with and predict vaccine-induced immunity are identified [http://dx.doi.org/10.1098/rstb.2014.0146].
Systems vaccinology has been successfully applied to a variety of licensed vaccines [yellow fever, influenza], and also to vaccine candidates against [HIV, malaria], resulting in the identification of early transcriptomic signatures that predict vaccine-induced antibody responses.

How to use sysvacc to inform better design (A systems framework for vaccine design Mooney2013), and how to move towards personalised vaccinology (https://doi.org/10.1016/j.vaccine.2017.07.062).

Overview, including pathogen-side factors

\subsection{Genetics factors affecting vaccine response}

Relatively few studies have assessed the impact of human genetic variation on responses[Franco, Lareau 2016].

This is despite evidence from genome-wide association studies suggesting such genetic variation influences immune response to vaccines and susceptibility to disease[Systems immunogenetics of vaccines.].
%     https://www.ncbi.nlm.nih.gov/pmc/articles/PMC3570049/
%     https://www.ncbi.nlm.nih.gov/pmc/articles/PMC5548390/
%     https://www.ncbi.nlm.nih.gov/pmc/articles/PMC2610683/

Search for "variation in vaccine response genetics GA Poland" in google scholar

Genetics of adverse events e.g. \url{https://www.ncbi.nlm.nih.gov/pubmed/18454680}

Results from vaccine-related twin studies e.g. in "TWIN STUDIES ON GENETIC VARIATIONS IN RESISTANCE TO TUBERCULOSIS", and (Defective T Memory Cell Differentiation after Varicella Zoster Vaccination in Older Individuals)

Review paper on GWAS for vaccines mooney2013SystemsImmunogeneticsVaccines

\section{Immune response to biologic therapies}

\section{Thesis overview}
% TODO

% By chapter CCC overview.
Chapters 1 and 2.
Chapter 3.
Chapter 4.
Chapter 5.
