%
% Chapter 1
%
% Unifies results chapters under the framework of human genetics.

\chapter{Introduction}

\section{Why study human genetics?}

Systems biology:

Linkage, GWAS (10 years), burden tests

Candidate genes (Border et al., 2019)

\enquote{molecular QTLs}
 
\todo{Add headings}

\gls{eQTL}
\gls{eQTL}
\glspl{eQTL}
\Glspl{eQTL}
%
coloc, fine mapping

TWAS, PheWAS\autocite{verma2017CurrentScopeChallenges}, MR

pathway analysis

Poster child gwas-eqtl-drug target

\section{The genetics of immune healthy}

Innate immunity is germline encoded

General genetics of immune response, including disease e.g. Immune Response Mainly Heritable? 

\section{Infectious diseases}

\section{Controlled immune perturbation}

\section{The genetics of immune response to perturbation}

Innate immunity is germline encoded

General genetics of immune response, including disease e.g. Immune Response Mainly Heritable? 

\section{Host factors affecting vaccine response}

Overview, including pathogen-side factors

Review of systems vaccinology

How to use sysvacc to inform better design (A systems framework for vaccine design Mooney2013), and how to move towards personalised vaccinology (https://doi.org/10.1016/j.vaccine.2017.07.062).

\section{The genetics of vaccine response}

Search for "variation in vaccine response genetics GA Poland" in google scholar

Genetics of adverse events e.g. \url{https://www.ncbi.nlm.nih.gov/pubmed/18454680}

Results from vaccine-related twin studies e.g. in "TWIN STUDIES ON GENETIC VARIATIONS IN RESISTANCE TO TUBERCULOSIS", and (Defective T Memory Cell Differentiation after Varicella Zoster Vaccination in Older Individuals)

Review paper on GWAS for vaccines mooney2013SystemsImmunogeneticsVaccines

\section{The genetics of drug response}

\section{Thesis overview}

By chapter overview of knowns and unknowns


