%
% Chapter 1
%
% Unifies results chapters under the framework of human genetics.

\chapter{Introduction}

\begin{outline}
    \1 Variation between humans exists
    \1 The eternal debate: nature vs nurture
    \1 Why study human genetics?
    \1 The structure of the genome and it's variation
    \1 Finding causal anchors
    \1 Leveraging natural G variation.
        % https://www.annualreviews.org/doi/pdf/10.1146/annurev-biodatasci-072018-021148
    % Genetic studies inform therapeutic discoveries and development
\end{outline}

\section{A brief overview of genetic association for complex traits}

\subsection{Early days}

\begin{outline}
    % From R.A. Fisher’s 1918 Paper to GWAS a Century Later
    \1 Early days, prior to GWAS
    \1 Mendelian genetics, family and linkage studies
    \1 Complex traits and the Common disease, common variant hypothesis
    \1 Twin studies and heritability estimates of complex traits
    % candidate_gene_studies folder in dloads
    \1 Candidate gene studies (Border et al., 2019)
    \1 Appreciation of polygenicity
\end{outline}

\subsection{The advent of \glsfmtshort{GWAS}}

\begin{outline}

    \1 10 years of GWAS
    \1 "The case of the missing heritability"

    \1 genotyping arrays
        \2 common known variants
        \2 designed to cover tag variants that represent most genetic variation
        \2 imputation

    \1 if discovering new var
    \1 WES (about 40Mbp of the genome)
        \2 covers more of the genome in terms of bp
        \2 but lower n, so lower power than array genotyping to do single variant associations
        \2 why 50x? variable coverage due to pulldown 

    \1 WGS
        \2 tradeoff between variant capture (n needed to observe variant) and sequencing depth (gives confidence to call variants)
        \2 20x ok to call 90\% of singletons

        \2 rare variants, including in nc regions
            \3 current discovery biases, finding higher effect size vars first

            \3 burden tests (e.g. SAIGE)
                \4 to get gene, aggregate based on variant consequence scores e.g. vep scores

        \2 structural variants

\end{outline}

\subsection{Narrowing the signal}

\begin{outline}

\1 PheWAS\autocite{verma2017CurrentScopeChallenges}

\1 Fine-mapping

    \2 as sample sizes get larger, and provided that sequencing or imputation can more exhaustively identify all of the candidate SNPs on the haplotype, rare recombination events will pile up, helping to make the causal SNP stand out above the passenger SNPs that usually travel on its haplotype [Huang 2017].

    \2 tag snps: causal snps may not be directly typed, may need to be imputed

\end{outline}

\subsection{Interpretation of genetic associations with molecular studies}

\begin{outline}

\1 Locus to gene mapping problem
    \2 nc snps
        \3 Genome-wide association studies have successfully identified genetic variants associated with immune-mediated disease, the majority of which are non-coding[10 Years of GWAS Discovery].

\1 using intermediate/endophenotypes
    \2 endophenotypes paper
    \2 expression as an important intermediate
        \3 measure by array, rnaseq 
    \2 theory is that genetic variants manifest their effects through these phenotypes, central dogma based

    \1 eqtl reviwe: albert2015RoleRegulatoryVariationa
    % and
    % Imprialou, M., Petretto, E., & Bottolo, L. (2017). Expression QTLs Mapping and Analysis: A Bayesian Perspective. In K. Schughart & R. W. Williams (Eds.), Systems Genetics (Vol. 1488, pp. 189–215). Springer New York. https://doi.org/10.1007/978-1-4939-6427-7_8

\1 coloc methods
    % (that photo on all the coloc methods that all attempt to solve the problem)

    \2 coloc
        \3 Under the assumption that the mechanism by which non-coding associations affect disease risk is through their effect on gene expression, a successful way to link associations to their target gene is by statistical colocalisation with eQTL datasets, to determine if the GWAS and eQTL signal share the same causal variant[Co-localization of Conditional eQTL and GWAS Signatures in Schizophrenia].

    \2 TWAS
    % When a TWAS gene also colocalizes with your trait of interest you have a 19% chance you've found the correct causal gene.
    % https://twitter.com/Eric_Fauman/status/1220885592193499137?s=09

    \2 MR

    \2 a transcriptional risk score (TRS)

\1 for eqtls, closest gene is often not the best candidate
        % https://twitter.com/Eric_Fauman/status/1198595609013489670/photo/1
    \2 annotation of nc var is functional genomics
        \3 e.g. gtex, ENCODE
        % http://www.cureffi.org/2019/12/04/prnp-gwas-other-traits/
        % Another solution is functional genomics — you ask whether the GWAS hit SNPs are also known, from other datasets, to have functional effects on a gene. See, as one example, a recent study of drug targets for immune-related traits [Fang 2019].
\end{outline}

\subsection{So what? Translational directions [can cut this whole section]}

\begin{outline}

\1 Why care?

    \2 polygenic scores, prs: marker for diagnosis

        \3 use in the clinic
            % doi: http://dx.doi.org/10.1101/19013086
            \4 e.g. polygenic background can modify penetrance 

        \3 but challenges from:

            \4 ancestry effects
            % https://twitter.com/skathire/status/1202554709107712001?s=09
            \4 need expanding into global populations, global biobanks e.g. Gains from Africa H3Africa, japanese biobanks

            % Variable prediction accuracy of polygenic scores within an ancestry group
            % https://elifesciences.org/articles/48376
            \4 non-ancestry effects

    \2 pathway analysis: "the great hairball gambit"

    \2 pathway prs
        \3 challenge is variant to gene assignment/mapping
            \4 e.g. restrictions to fine mapped eQTLs

    \2 Understand mech. of causal genes: molecular pathogenesis

    \2 Drug target prioritisation for disease traits
    % higher success rate
    % https://doi.org/10.1371/journal.pgen.1008489

    \2 how to drug a complex disease with no single 'candidate gene'?

        \3 e.g. of successful GWAS -> drug target
            \4 drug targets with genetic support are more likely

        \3 building allelic series

% GWASing and fine-mapping complex diseases like IBD turns out a large number of common causal variants with small-effect sizes.
% - Is polygenicity a population or individual property? i.e. are most individual IBD cases driven solely by a distribution of small-effects, or do most patients also have 1 or more large-effect rare variants that point out priority targets for their own personalised treatment?
% - Do many of these common causal variants e.g. converge to hit on the same pathways?
% - Otherwise, what is the use of these target discovery pipelines that output ranked lists of target genes? Could a drug designed to modulate a single protein target be expected to work for a large number of patients?

\end{outline}

\section{The effects of genetic variation on expression: context is key}

\begin{outline}

\1 inthe dreaded GxE interaction
    \2 "In genetics, context matters"
    % https://www.nature.com/articles/s41588-019-0515-7

    \2 for both gwas, and molQTLs, context is key

\1 Architecture varies e.g. across cell type and tissues
% Common regulatory variation impacts gene expression in a cell type–dependent manner
% The architecture of gene regulatory variation across multiple human tissues: The MuTHER study.
    \2 tissue
    \2 cell type
    \2 interaction between cells in vivo
    \2 stimulation conditions

    % https://academic.oup.com/hmg/article/23/7/1947/655184/
\1 QTLs can interact with sex and age

\1 types of context specific QTL
    \2 ackerman conditional vs dynamic

\1 Mechanisms of reQTLs
What molecular mechanisms might allow for interaction between \Gls{eQTL} and different environmental conditions?
% Fu, J., Wolfs, M. G. M., Deelen, P., Westra, H.-J., Fehrmann, R. S. N., te Meerman, G. J., … Franke, L. (2012). Unraveling the Regulatory Mechanisms Underlying Tissue-Dependent Genetic Variation of Gene Expression. PLoS Genetics, 8(1), e1002431. https://doi.org/10.1371/journal.pgen.1002431
Four categories of tissue-dependent cis-eQTL effects, and proposed two molecular models.

% Rotival, M. (2019). Characterising the genetic basis of immune response variation to identify causal mechanisms underlying disease susceptibility. HLA, 94(3), 275?284. https://doi.org/10.1111/tan.13598

% notice that reQTLs and DE can be decoupled.
% Maranville, J. C., Luca, F., Richards, A. L., Wen, X., Witonsky, D. B., Baxter, S., Stephens, M., & Di Rienzo, A. (2011). Interactions between Glucocorticoid Treatment and Cis-Regulatory Polymorphisms Contribute to Cellular Response Phenotypes. PLoS Genetics, 7(7), e1002162. https://doi.org/10.1371/journal.pgen.1002162

coloc of immune mediated traits is enhanced by context-specific eQTLs

\end{outline}

\section{Immunity is a complex trait}

Is it even plausible that genetic var is important?
Brodin: most env paper.

Immune-mediated diseases
Heritability of immune parameters and immune-mediated diseases
    ranges from 

\subsection{Genetic factors affecting the healthy immune system}

Why study health?
Factors affecting the healthy immune system.

In healthy populations, $\approx$50\% variation in immune system driven by non-genetic factors, $\approx$30–40\% variation is driven by genetic variation (Liston and Goris 2018).

"Such systems  immunology studies in  healthy individuals have  revealed that human immune  systems are incredibly  variable among individuals,  but very stable within  individuals over time (11),  and most of this variation is  attributed to non-heritable  factors (12)."

\subsection{Genetic factors affecting immune response to challenge}

Given the genetic control of the healthy immune system, one can hypothesise that immune response to challenge may also be influenced by genetic factors.

The need for controlled immune challenge in trials.
Studies of natural infection are complicated.
clinical trials as an opportunity: 
    Vaccines and drugs used as controlled immune challenge.

Posit that eQTls where the genetic effect of 

\subsubsection{Context-specific immune response eQTLs in vitro}

The majority of response eQTL mapping experiments to date have been conducted \textit{in vitro}, where one can precisely adjust both the length and intensity of stimulation.
Environmental variables including cell type composition or tissue type that are expected to interact with the eQTL effect and may confound the interaction effect with stimulation can be controlled.
The choice of experiment system and stimulation can also be hypothesis-driven, for example, if certain tissues are expected to be more relevant for a specific disease. 
\todo{add more pros for in vitro reQTLs here, and find citations}.

One of the first studies to perform \gls{reQTL} mapping for an immune stimulation was \autocite{barreiro2012DecipheringGeneticArchitecture}, where eQTLs were mapped separately in monocyte-derived dendritic cells before and after 18h infection with \textit{Mycobacterium tuberculosis}.
reQTLs were detected for 198 genes, 102 specific to the uninfected state, and 96 specific to the infected state. 
These reQTLs were enriched for GWAS SNPs associated with host susceptibility to tuberculosis; this was not observed for eQTLs that were not reQTLs.

Since then, \textit{in vitro} immune reQTL studies have been conducted for a variety of experimental systems (e.g. primary CD14+ monocytes\autocite{fairfax2014InnateImmuneActivity}) and stimulations (IFN$\gamma$ and LPS\autocite{fairfax2014GeneticsGeneExpression}).

Take home messages:
- reQTLs develop trans-effects on stimulation \autocite{fairfax2014InnateImmuneActivity}
Overall, as the number of experimental systems and stimulations increases, large number of eQTLs are only detected.

moost recent are very high thrgouhpug
% https://www.ncbi.nlm.nih.gov/pmc/articles/PMC5131815/

\subsubsection{\textit{in vivo} response QTL mapping}

less popular
A complementary approach.

in vivo pros
    choice of context 
    whole organism phenotypes
    more likely to be repeated measures

Review of in vivo mapping.
What we learn on top of in vitro
(Franco et al., 2013)
% TB https://www.nature.com/articles/srep16882
% Franco
% Lareau smallpox apoptosis
% Caliskan Rhinovirus
% Davenport

Large cohorts: 
% vQTLs possible: https://advances.sciencemag.org/content/5/8/eaaw3538/tab-pdf

\section{Immune response to vaccination}

Vaccination has enormous impact on global health [10.1098/rstb.2013.0433].

Vaccines stimulate the immune system with pathogen-derived antigens to induce effector responses (primarily antigen-specific antibodies) and immunological memory against the pathogen itself.
These effector responses are then be rapidly reactivated in cases of future exposure to the pathogen, mediating long-term protection.

\subsection{Systems vaccinology: from empirical to rational vaccinology}

History of vaccine dev
[summary of low-throughput immunology e.g. animal models]  

- Vaccination coverage in vunerable populations is below optimal

However, a vaccine that is highly efficacious in one human population may have significantly lower efficacy in other populations.
[1 statistic on vaccine efficiacy differences e.g. rotavirus]
% Vaccination of special populations: Protecting the vulnerable.
Particularly challenging populations for vaccination include the infants and elderly, pregnant, immuno comprimised patients, ethnically-diverse populations, and developing countries.
For the majority of licensed vaccines, there is a lack of understanding regarding the molecular mechanisms that underpin this variation in host immune response.
Immunological mechanisms that underpin a specific vaccine's success or failure in a given individual are often poorly understood[Immunological mechanisms of vaccination]. 

rational vacc, where the key is sys vacc

Review of systems vaccinology (pull out of self\_viva\_copypasta)
These systems vaccinology studies often consider longitudinal measurements of the transcriptomic, cellular, cytokine, and antibody immune responses following vaccination[Vaccinology in the era of high-throughput biology.].

Systems vaccinology is the application of -omics technologies to provide a systems-level characterisation of the human immune system after vaccine-perturbation.
Measurements are taken at multiple molecular levels (e.g. genome, transcriptome, proteome), and molecular signatures that correlate with and predict vaccine-induced immunity are identified [http://dx.doi.org/10.1098/rstb.2014.0146].
\todo{define what a signature is}
Systems vaccinology has been successfully applied to a variety of licensed vaccines [yellow fever, influenza], and also to vaccine candidates against [HIV, malaria], resulting in the identification of early transcriptomic signatures that predict vaccine-induced antibody responses.

% Merge in first year report intro

Cotugno
- dna meth: DNA methylation [52, 53, 54] events

How to use sysvacc to inform better design (A systems framework for vaccine design Mooney2013), and how to move towards personalised vaccinology (https://doi.org/10.1016/j.vaccine.2017.07.062).

Overview, including pathogen-side factors

\subsection{Genetic factors affecting vaccine response}

Read this \url{Vaccine. 2018 August 28; 36(36): 5350–5357. doi:10.1016/j.vaccine.2017.07.062.}
Search for "variation in vaccine response genetics GA Poland" in google scholar

% https://www.thelancet.com/journals/laninf/article/PIIS1473-3099(19)30121-5/fulltext

% See protocol paper for references

measles
% https://www.sciencedaily.com/releases/2011/09/110922134546.htm

Relatively few studies have assessed the impact of human genetic variation on responses[Franco, Lareau 2016].
% Franco to be touched on in chapter 3

This is despite evidence from genome-wide association studies suggesting such genetic variation influences immune response to vaccines and susceptibility to disease[Systems immunogenetics of vaccines.].
%     https://www.ncbi.nlm.nih.gov/pmc/articles/PMC3570049/
%     https://www.ncbi.nlm.nih.gov/pmc/articles/PMC5548390/
%     https://www.ncbi.nlm.nih.gov/pmc/articles/PMC2610683/

Results from vaccine-related twin studies e.g. in "TWIN STUDIES ON GENETIC VARIATIONS IN RESISTANCE TO TUBERCULOSIS", and (Defective T Memory Cell Differentiation after Varicella Zoster Vaccination in Older Individuals)

Nice summary table for gwas 
Review paper on GWAS for vaccines mooney2013SystemsImmunogeneticsVaccines

Genetics of adverse events e.g. \url{https://www.ncbi.nlm.nih.gov/pubmed/18454680}

\section{Immune response to biologic therapies}

\subsection{Genetic factors affecting biologic responses}

e.g. PANTS immunogenicity
% touched on in chapter 3

\section{Thesis overview}

% By chapter CCC overview.
Chapters 1 and 2.
Chapter 3.
Chapter 4.
Chapter 5.
