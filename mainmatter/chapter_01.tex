%
% Chapter 1
%
% Unifies results chapters under the framework of human genetics.

\chapter{Introduction}

\section{Why study human genetics?}
Causal anchors.
Leveraging natural variation.

\section{A brief history of complex trait genetics}

Mendellian genetics, family and linkage studies

Common variation, X years of GWAS
Candidate gene studies (Border et al., 2019)
Missing heritability

Rare variation, burden tests

Locus to gene problem, nc variation.
Genome-wide association studies have successfully identified genetic variants associated with immune-mediated disease, the majority of which are non-coding[10 Years of GWAS Discovery].

Under the assumption that the mechanism by which non-coding associations affect disease risk is through their effect on gene expression, a successful way to link associations to their target gene is by statistical colocalisation with eQTL datasets, to determine if the GWAS and eQTL signal share the same causal variant[Co-localization of Conditional eQTL and GWAS Signatures in Schizophrenia].

    Fine-mapping
    Coloc
    Pathway analysis
    molecular \glspl{QTL}
    % Many types (review)
    TWAS, PheWAS\autocite{verma2017CurrentScopeChallenges}, MR

Drug target prioritisation for disease traits
e.g. of successful GWAS -> drug target

\section{Immunity is a complex trait}
Immune-mediated diseases
Heritability of immune parameters and immune-mediated diseases

\subsection{Genetic factors affecting the healthy immune system}

Factors affecting the healthy immune system.

\subsection{Genetic factors affecting immune response to challenge}

Given the genetic control of the healthy immune system, one can hypothesise that immune response to challenge may also be influenced by genetic factors.

\section{Vaccines for controlled immune challenge}

One issue is controlling perturbation.

\subsection{Empirical to systems vaccinology}

Vaccines have enormous impact on global health and quality of life, but the immunological mechanisms that underpin a specific vaccine’s success or failure in a given individual are often poorly understood[Immunological mechanisms of vaccination]. 

Omics technologies are increasingly employed to model the factors that cause variation in individual vaccination outcome on a systems level. 
These systems vaccinology studies often consider longitudinal measurements of the transcriptomic, cellular, cytokine, and antibody immune responses following vaccination[Vaccinology in the era of high-throughput biology.].

Relatively few studies have assessed the impact of human genetic variation on responses[Franco, Lareau 2016].

This is despite evidence from genome-wide association studies suggesting such genetic variation influences immune response to vaccines and susceptibility to disease[Systems immunogenetics of vaccines.].

Overview, including pathogen-side factors

Review of systems vaccinology (pull out of self\_viva\_copypasta)

    How to use sysvacc to inform better design (A systems framework for vaccine design Mooney2013), and how to move towards personalised vaccinology (https://doi.org/10.1016/j.vaccine.2017.07.062).

\subsection{The genetics of vaccine response}

Search for "variation in vaccine response genetics GA Poland" in google scholar

Genetics of adverse events e.g. \url{https://www.ncbi.nlm.nih.gov/pubmed/18454680}

Results from vaccine-related twin studies e.g. in "TWIN STUDIES ON GENETIC VARIATIONS IN RESISTANCE TO TUBERCULOSIS", and (Defective T Memory Cell Differentiation after Varicella Zoster Vaccination in Older Individuals)

Review paper on GWAS for vaccines mooney2013SystemsImmunogeneticsVaccines

\section{Drugs for controlled immune challenge}

\section{Thesis overview}

By chapter CCC overview.
