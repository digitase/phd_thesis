%
% Chapter 2
%

\chapter{Transcriptomic response to pandemic influenza H1N1/09 vaccine (Pandemrix)}

\section{Introduction}

\subsection{<H1N1 virus and vaccine biology>}

Intro to the virus (Characteristics of Swine-Origin 2009 A(H1N1), DOI: 10.1126/science.1176225)

Pandemrix, as one of several vaccines licensed: \url{https://www.ema.europa.eu/en/human-regulatory/overview/public-health-threats/pandemic-influenza/2009-h1n1-influenza-pandemic/medicines-authorised-during-pandemic}

Relationship to seasonal H1N1.
...a single dose of monovalent 2009 H1N1 vaccine was recommended in adults, but
young children were recommended to receive 2 doses (reviewed by [3••]). It is
likely that a single dose was sufficient to induce immunity in adults because
prior exposure to seasonal H1N1 viruses had immunologically primed the
population. \url{https://www.ncbi.nlm.nih.gov/pmc/articles/PMC3224079/}
"Seasonal influenza vaccine provides priming for A/H1N1 immunization." \url{https://www.ncbi.nlm.nih.gov/pubmed/20371459}
Demonstration in a mouse model: \url{https://www.ncbi.nlm.nih.gov/pmc/articles/PMC3024675/}

- more variation will be explained by history of exposure rather than genetics, so may be harder to detect

\section{Methods}

\subsection{The Human Immune Response Dynamics (HIRD) cohort}

\subsection{RNAseq}

Do we have enough reads for RNAseq analysis? \url{https://www.ncbi.nlm.nih.gov/pubmed/24434847} and doi:10.1093/bioinformatics/btt688

\subsection{DGE}

Batch effect correction (see batch effects tag)
% Chen, C., Grennan, K., Badner, J., Zhang, D., Gershon, E., Jin, L., & Liu, C. (2011). Removing Batch Effects in Analysis of Expression Microarray Data: An Evaluation of Six Batch Adjustment Methods. PLoS ONE, 6(2), e17238. https://doi.org/10.1371/journal.pone.0017238
Combat is best here
% Espín-Pérez, A., Portier, C., Chadeau-Hyam, M., van Veldhoven, K., Kleinjans, J. C. S., & de Kok, T. M. C. M. (2018). Comparison of statistical methods and the use of quality control samples for batch effect correction in human transcriptome data. PLOS ONE, 13(8), e0202947. https://doi.org/10.1371/journal.pone.0202947
LM, LMM, Combat were comparable
% Liu, Q., & Markatou, M. (2016). Evaluation of Methods in Removing Batch Effects on RNA-seq Data. Infectious Diseases and Translational Medicine, 2(1), 3–9. https://doi.org/10.11979/idtm.201601002
In some cases, Combat overcorrects
% Nygaard, V., Rødland, E. A., & Hovig, E. (2015). Methods that remove batch effects while retaining group differences may lead to exaggerated confidence in downstream analyses. Biostatistics, (January), kxv027. https://doi.org/10.1093/biostatistics/kxv027
But main issue is unbalanced design, which affects even 2-way anova. Rather than 2-step, Safest is to use a covariate, which seems to at least create appropriate conficence intervals (1e)


Why combine -7 and 0? See Sobolev:
(a) Observed values of multivariate statistic t (m.v.t.) quantifying global PBMC gene-expression dissimilarity in comparison of two study days (red dots) to values expected when days are randomly assigned between groups.

\subsection{Meta-analysis}

Whilst there is a slew of literature on metanalyasis of rnaseq adn array (e.g. metaMA), combining platforms is fraught with difficultiy.
different probes, 
different tech -> diff stat models

Why expected het?
platform effect (ratio compression, differences in preprocessing to genes). different sets of samples (more extreme in array)

exmplaes of 
e.g. random effects model of approx 24 datasets: https://academic.oup.com/nar/article/45/17/9860/4084660#106485896
e.g. sva: https://www.ncbi.nlm.nih.gov/pmc/articles/PMC3617154/


Alternative is: CorMotif first applies limma (Smyth, 2004) to each study separately.
CorMotif for microarray data since it was motivated by the microarray analysis in the SHH study. However, the idea behind CorMotif is general, and it should be straightforward to develop a similar framework for RNA-seq data.

Or MetaVolcano: vote counting, REM (note small k), 

Sweeny tests diff ks: Methods to increase reproducibility in differential

or complicated R package, CBM (“Cross-platform Bayesian Model”),
also see CBM paper for discussion of difficulties of combinding platform

cannot acually use CBM, as it operates on expressions, with a binary case vs control, so no covariates
same limitation for cormotif, although it takes any number of groups

rankprod, mayday seasight

\subsubsection{Choice of meta-analysis method}


% Gonnermann, A., Framke, T., Großhennig, A., & Koch, A. (2015). No solution yet for combining two independent studies in the presence of heterogeneity: COMMENTARY. Statistics in Medicine, 34(16), 2476–2480. https://doi.org/10.1002/sim.6473
The problem

Although expected to differ bt gene,
Most estimates of tau 0

Differs based on scale of paramters
Differs based on effect


% https://github.com/stan-dev/stan/wiki/Prior-Choice-Recommendations
Recommendations from Gelman may be too weak.


But we have empirical distribution

% Higgins, J. P. T., Thompson, S. G., & Spiegelhalter, D. J. (2009). A re-evaluation of random-effects meta-analysis. Journal of the Royal Statistical Society: Series A (Statistics in Society), 172(1), 137–159. https://doi.org/10.1111/j.1467-985X.2008.00552.x
Similar to Higgins: an informative prior distribution fit on observed hets


use MLE fitdistr
    empirical bayes is popular for high dim data. esp EB for sharing e.g. limmavoom, combat(method of moments)



\autocite{Sobolev2016}

\section{Results}

\subsection{Comparison to Sobolev R vs. NR}

Stuff from 1st year report.

