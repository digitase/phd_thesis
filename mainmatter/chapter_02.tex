%
% Chapter 2
%
% Systems vacc-style paper

\chapter{Transcriptomic response to pandemic influenza H1N1/09 vaccine (Pandemrix)}

\section{Introduction}

\subsection{Response to inactivated influenza vaccines}

\subsection{The H1N1 virus, Pandemrix, and Pandemrix response}

Intro to the virus (Characteristics of Swine-Origin 2009 A(H1N1), DOI: 10.1126/science.1176225)

Pandemrix, as one of several vaccines licensed: \url{https://www.ema.europa.eu/en/human-regulatory/overview/public-health-threats/pandemic-influenza/2009-h1n1-influenza-pandemic/medicines-authorised-during-pandemic}

Relationship to seasonal H1N1.
...a single dose of monovalent 2009 H1N1 vaccine was recommended in adults, but
young children were recommended to receive 2 doses (reviewed by [3••]). It is
likely that a single dose was sufficient to induce immunity in adults because
prior exposure to seasonal H1N1 viruses had immunologically primed the
population. \url{https://www.ncbi.nlm.nih.gov/pmc/articles/PMC3224079/}
"Seasonal influenza vaccine provides priming for A/H1N1 immunization." \url{https://www.ncbi.nlm.nih.gov/pubmed/20371459}
Demonstration in a mouse model: \url{https://www.ncbi.nlm.nih.gov/pmc/articles/PMC3024675/}

- more variation will be explained by history of exposure rather than genetics, so may be harder to detect

\subsection{Response to AS03}

\subsection{the narcolepsy controversy}

Prevacc signatures of Tri
Using larger transcriptomic dataset
Are they genetic

\section{Methods}

\subsection{The Human Immune Response Dynamics (HIRD) cohort}

\subsection{TRI}

% Nauta JJ, Beyer WE, Osterhaus AD. On the relationship between mean antibody level, seroprotection and clinical protection from influenza. Biologicals 2009; 37:216-21; PMID:19268607; http://dx. doi.org/10.1016/j.biologicals.2009.02.002
In clinical studies seroprotection is normally defined as a specific antibody titer or antibody titer increase (seroconversion).22

\subsection{RNAseq}

Do we have enough reads for RNAseq analysis? \url{https://www.ncbi.nlm.nih.gov/pubmed/24434847} and doi:10.1093/bioinformatics/btt688

\subsection{DGE}

Batch effect correction (see batch effects tag)
% Chen, C., Grennan, K., Badner, J., Zhang, D., Gershon, E., Jin, L., & Liu, C. (2011). Removing Batch Effects in Analysis of Expression Microarray Data: An Evaluation of Six Batch Adjustment Methods. PLoS ONE, 6(2), e17238. https://doi.org/10.1371/journal.pone.0017238
Combat is best here
% Espín-Pérez, A., Portier, C., Chadeau-Hyam, M., van Veldhoven, K., Kleinjans, J. C. S., & de Kok, T. M. C. M. (2018). Comparison of statistical methods and the use of quality control samples for batch effect correction in human transcriptome data. PLOS ONE, 13(8), e0202947. https://doi.org/10.1371/journal.pone.0202947
LM, LMM, Combat were comparable
% Liu, Q., & Markatou, M. (2016). Evaluation of Methods in Removing Batch Effects on RNA-seq Data. Infectious Diseases and Translational Medicine, 2(1), 3–9. https://doi.org/10.11979/idtm.201601002
In some cases, Combat overcorrects
% Nygaard, V., Rødland, E. A., & Hovig, E. (2015). Methods that remove batch effects while retaining group differences may lead to exaggerated confidence in downstream analyses. Biostatistics, (January), kxv027. https://doi.org/10.1093/biostatistics/kxv027
But main issue is unbalanced design, which affects even 2-way anova. Rather than 2-step, Safest is to use a covariate, which seems to at least create appropriate conficence intervals (1e)


Why combine -7 and 0? See Sobolev:
(a) Observed values of multivariate statistic t (m.v.t.) quantifying global PBMC gene-expression dissimilarity in comparison of two study days (red dots) to values expected when days are randomly assigned between groups.

Should we meta-analyse?

% https://journals.plos.org/plosone/article?id=10.1371/journal.pone.0059202
In conclusion, we found that underpowered studies play a very substantial role in meta-analyses reported by Cochrane reviews, since the majority of meta-analyses include no adequately powered studies. In meta-analyses including two or more adequately powered studies, the remaining underpowered studies often contributed little information to the combined results, and could be left out if a rapid review of the evidence is required.

\subsection{Meta-analysis}

Whilst there is a slew of literature on metanalyasis of rnaseq adn array (e.g. metaMA), combining platforms is fraught with difficultiy.
different probes, 
different tech -> diff stat models

Why expected het?
platform effect (ratio compression, differences in preprocessing to genes). different sets of samples (more extreme in array)

exmplaes of 
e.g. random effects model of approx 24 datasets: 
% https://academic.oup.com/nar/article/45/17/9860/4084660#106485896
e.g. sva: https://www.ncbi.nlm.nih.gov/pmc/articles/PMC3617154/


Alternative is: CorMotif first applies limma (Smyth, 2004) to each study separately.
CorMotif for microarray data since it was motivated by the microarray analysis in the SHH study. However, the idea behind CorMotif is general, and it should be straightforward to develop a similar framework for RNA-seq data.

Or MetaVolcano: vote counting, REM (note small k), 

Sweeny tests diff ks: Methods to increase reproducibility in differential

or complicated R package, CBM (“Cross-platform Bayesian Model”),
also see CBM paper for discussion of difficulties of combinding platform

cannot acually use CBM, as it operates on expressions, with a binary case vs control, so no covariates
same limitation for cormotif, although it takes any number of groups

rankprod (focus on case/control design), mayday seasight

\subsubsection{Choice of meta-analysis method}

Two schools of thought for frequentist meta-analysis: 
fixed-effect, 
% Higgins, J. P. T., Thompson, S. G., & Spiegelhalter, D. J. (2009). A re-evaluation of random-effects meta-analysis. Journal of the Royal Statistical Society: Series A (Statistics in Society), 172(1), 137–159. https://doi.org/10.1111/j.1467-985X.2008.00552.x
or in the presence of het, random-effects.

We have het, so def use random effects.

% Veroniki, A. A., Jackson, D., Viechtbauer, W., Bender, R., Bowden, J., Knapp, G., … Salanti, G. (2016). Methods to estimate the between-study variance and its uncertainty in meta-analysis. Research Synthesis Methods, 7(1), 55–79. https://doi.org/10.1002/jrsm.1164
How to estimate het?
Many methods to estimate het, but
% Gonnermann, A., Framke, T., Großhennig, A., & Koch, A. (2015). No solution yet for combining two independent studies in the presence of heterogeneity: COMMENTARY. Statistics in Medicine, 34(16), 2476–2480. https://doi.org/10.1002/sim.6473
% Friede, T., Röver, C., Wandel, S., & Neuenschwander, B. (2017). Meta-analysis of few small studies in orphan diseases: Meta-Analysis of Few Small Studies. Research Synthesis Methods, 8(1), 79–91. https://doi.org/10.1002/jrsm.1217
% Friede, T., Röver, C., Wandel, S., & Neuenschwander, B. (2017). Meta-analysis of two studies in the presence of heterogeneity with applications in rare diseases: Meta-analysis of two studies in the presence of heterogeneity. Biometrical Journal, 59(4), 658–671. https://doi.org/10.1002/bimj.201500236
% Seide, S. E., Röver, C., & Friede, T. (2019). Likelihood-based random-effects meta-analysis with few studies: Empirical and simulation studies. BMC Medical Research Methodology, 19(1). https://doi.org/10.1186/s12874-018-0618-3

The problem: we only have k=2, and MLE estimates of tau are not very good with k=2.
Highly imprecise, and often:
% Chung, Y., Rabe-Hesketh, S., Dorie, V., Gelman, A., & Liu, J. (2013). A Nondegenerate Penalized Likelihood Estimator for Variance Parameters in Multilevel Models. Psychometrika, 78(4), 685–709. https://doi.org/10.1007/s11336-013-9328-2
boundary estimate problems.  

and We know 0 het is inappropriate.

% Friede, T., Röver, C., Wandel, S., & Neuenschwander, B. (2017). Meta-analysis of few small studies in orphan diseases: Meta-Analysis of Few Small Studies. Research Synthesis Methods, 8(1), 79–91. https://doi.org/10.1002/jrsm.1217
% Friede, T., Röver, C., Wandel, S., & Neuenschwander, B. (2017). Meta-analysis of two studies in the presence of heterogeneity with applications in rare diseases: Meta-analysis of two studies in the presence of heterogeneity. Biometrical Journal, 59(4), 658–671. https://doi.org/10.1002/bimj.201500236
% Seide, S. E., Röver, C., & Friede, T. (2019). Likelihood-based random-effects meta-analysis with few studies: Empirical and simulation studies. BMC Medical Research Methodology, 19(1). https://doi.org/10.1186/s12874-018-0618-3
Bayesian random-effects meta is attractive.
But What priors should we use?

Prior for tau.

% Gelman (2006)
A general rec is:
Use distribution in the half-t family e.g. Cauchy (df=1) when the number of groups is small and in other settings where a weakly-informative prior is desired.
In their 3-schools examples, choose a value of scale just higher than expected, this is to weakly constrain the posterior, and not to actually represent prior knowledge.
- Warn against inverse-gamma(e, e), as it can influence the posterior mean.

% Friede, T., Röver, C., Wandel, S., & Neuenschwander, B. (2017). Meta-analysis of few small studies in orphan diseases: Meta-Analysis of Few Small Studies. Research Synthesis Methods, 8(1), 79–91. https://doi.org/10.1002/jrsm.1217
% Friede, T., Röver, C., Wandel, S., & Neuenschwander, B. (2017). Meta-analysis of two studies in the presence of heterogeneity with applications in rare diseases: Meta-analysis of two studies in the presence of heterogeneity. Biometrical Journal, 59(4), 658–671. https://doi.org/10.1002/bimj.201500236
% Seide, S. E., Röver, C., & Friede, T. (2019). Likelihood-based random-effects meta-analysis with few studies: Empirical and simulation studies. BMC Medical Research Methodology, 19(1). https://doi.org/10.1186/s12874-018-0618-3
But weak priors are not recommended, as k is small, so there is little information in the data.
% https://github.com/stan-dev/stan/wiki/Prior-Choice-Recommendations
% The Gelman (2006) half-cauchy may be too weak for small k.

We can get empirical distribution of many genes.
    % see this for a justification of why REML over ML
    % Veroniki, A. A., Jackson, D., Viechtbauer, W., Bender, R., Bowden, J., Knapp, G., … Salanti, G. (2016). Methods to estimate the between-study variance and its uncertainty in meta-analysis. Research Synthesis Methods, 7(1), 55–79. https://doi.org/10.1002/jrsm.1164
    fit a default reml model, exclude 0 ests.
    % By default, the starting value is set equal to the value of the Hedges (HE) estimator and the algorithm terminates when the change in the estimated value of  2 is smaller than 105 from one iteration to the next.
Advantage of getting the correct parameter scale for our data.
So use Empirical Bayes:
    aside: empirical bayes is popular for high dim data e.g. edgeR, DESeq2, limma-voom, combat (method of moments)

Papers that fit empirical datasets for tau2: Most of these are inverse-gamma/log-t family
% Higgins, J. P. T., & Whitehead, A. (1996). Borrowing strength from external trials in a meta-analysis. Statistics in Medicine, 15(24), 2733–2749. https://doi.org/10.1002/(SICI)1097-0258(19961230)15:24<2733::AID-SIM562>3.0.CO;2-0
Fit inverse gamma distribution on method of moments estimates from 18 gastroenterology trials with similar endpoints.
% Pullenayegum, E. M. (2011). An informed reference prior for between-study heterogeneity in meta-analyses of binary outcomes: Prior for between-study heterogeneity. Statistics in Medicine, 30(26), 3082–3094. https://doi.org/10.1002/sim.4326
This paper has described the distribution of the between-study variance amongst Cochrane reviews published between 2008 and 2009, and investigating a binary outcome. A log-normal distribution incorporating the association between the between-study variance and the pooled effect size gave the best fit.
% Turner, R. M., Davey, J., Clarke, M. J., Thompson, S. G., & Higgins, J. P. (2012). Predicting the extent of heterogeneity in meta-analysis, using empirical data from the Cochrane Database of Systematic Reviews. International Journal of Epidemiology, 41(3), 818–827. https://doi.org/10.1093/ije/dys041
Predictive distributions are presented for nine different settings, defined by type of outcome and type of intervention comparison. For example, for a planned meta-analysis comparing a pharmacological intervention against placebo or control with a subjectively measured outcome, the predictive distribution for heterogeneity is a log-normal (2.13, 1.582) distribution, which has a median value of 0.12.
% Rhodes, K. M., Turner, R. M., & Higgins, J. P. T. (2015). Predictive distributions were developed for the extent of heterogeneity in meta-analyses of continuous outcome data. Journal of Clinical Epidemiology, 68(1), 52–60. https://doi.org/10.1016/j.jclinepi.2014.08.012
Model selection based on the deviance information criterion (DIC) [8] led to the choice of the log-t model for t2. (5df)
% Turner, R. M., Jackson, D., Wei, Y., Thompson, S. G., & Higgins, J. P. T. (2015). Predictive distributions for between-study heterogeneity and simple methods for their application in Bayesian meta-analysis: R. M. TURNER ET AL. Statistics in Medicine, 34(6), 984–998. https://doi.org/10.1002/sim.6381
The priors are derived as log-normal distributions for the between-study variance, applicable to meta-analyses of binary outcomes on the log odds-ratio scale.
%

% Chung, Y., Rabe-Hesketh, S., Dorie, V., Gelman, A., & Liu, J. (2013). A Nondegenerate Penalized Likelihood Estimator for Variance Parameters in Multilevel Models. Psychometrika, 78(4), 685–709. https://doi.org/10.1007/s11336-013-9328-2
We choose gamma: as Density at tau=0 is 0, but increases linearly from 0, so values close to 0 are still permitted if the data suggests it.
For lognormal/inverse gamma, they have a derivative of 0 at tau=0, so they rule out small tau no matter what the data suggest.
For The exponential and half-Cauchy families, for example, do not decline to zero at the boundary, so they do not rule out posterior mode estimates of zero.

% Maximum-likelihood Fitting of Univariate Distributions
% https://stat.ethz.ch/R-manual/R-devel/library/MASS/html/fitdistr.html
% For the Normal, log-Normal, geometric, exponential and Poisson distributions the closed-form MLEs (and exact standard errors) are used, and start should not be supplied.
% For all other distributions, direct optimization of the log-likelihood is performed using optim.

Prior for logFC

Not as much discussion in the lit:
% Gelman (2006)
There is Typically enough data to estimate this to use a non informative prior.
% Friede, T., Röver, C., Wandel, S., & Neuenschwander, B. (2017). Meta-analysis of few small studies in orphan diseases: Meta-Analysis of Few Small Studies. Research Synthesis Methods, 8(1), 79–91. https://doi.org/10.1002/jrsm.1217
% Friede, T., Röver, C., Wandel, S., & Neuenschwander, B. (2017). Meta-analysis of two studies in the presence of heterogeneity with applications in rare diseases: Meta-analysis of two studies in the presence of heterogeneity. Biometrical Journal, 59(4), 658–671. https://doi.org/10.1002/bimj.201500236
Even Friede Uses noninformative flat.

Two choices in bayesmeta are uniform and normal.
We know Mean is 0: most genes are not DE.
so flat prior makes no sense

% Also: general principles point towards weaker priors, but not uniform????
% https://github.com/stan-dev/stan/wiki/Prior-Choice-Recommendations
% For example, it is common to expect realistic effect sizes to be of order of magnitude 0.1 on a standardized scale (for example, an educational innovation that might improve test scores by 0.1 standard deviations). In that case, a prior of N(0,1) could be considered very strong, in that it puts most of its mass on parameter values that are unrealistically large in absolute value. When we say this prior is "weakly informative," what we mean is that, if there's a reasonably large amount of data, the likelihood will dominate, and the prior will not be important. If the data are weak, though, this "weakly informative prior" will strongly influence the posterior inference. The phrase "weakly informative" is implicitly in comparison to a default flat prior.
% Weakly informative prior should contain enough information to regularize: the idea is that the prior rules out unreasonable parameter values but is not so strong as to rule out values that might make sense
% Weakly informative rather than fully informative: the idea is that the loss in precision by making the prior a bit too weak (compared to the true population distribution of parameters or the current expert state of knowledge) is less serious than the gain in robustness by including parts of parameter space that might be relevant. It's been hard for us to formalize this idea.
% One principle: write down what you think the prior should be, then spread it out. The idea is that the cost of setting the prior too narrow is more severe than the cost of setting it too wide. I've been having trouble formalizing this idea.
% Don't use uniform priors, or hard constraints more generally, unless the bounds
% represent true constraints (such as scale parameters being restricted to be
% positive, or correlations restricted to being between -1 and 1).

To avoid overshrinking, could consider heavy-tailed priors (e.g. cauchy) for mu rather than normal, but this is not possible in bayesmeta.
% Gelman, A., Jakulin, A., Pittau, M. G., & Su, Y.-S. (2008). A weakly informative default prior distribution for logistic and other regression models. The Annals of Applied Statistics, 2(4), 1360–1383. https://doi.org/10.1214/08-AOAS191
Cauchy 2.5
% Zhu, A., Ibrahim, J. G., & Love, M. I. (2018). Heavy-tailed prior distributions for sequence count data: Removing the noise and preserving large differences. BioRxiv. https://doi.org/10.1101/303255
DEseq/apeglm: prior on logfc, cauchy with scale adapted.

But bayesmeta is normal. So weaken further to place more prior on larger values.
This means less shrinkage.

Also: we will shrink again with ashr. 
which can fit a more complicated distr (mixture?)

% https://github.com/stan-dev/stan/wiki/Prior-Choice-Recommendations
So We use a very weak normal prior, scaled to each coef, as we still want some scaling based on parameter scales.
Equiv to saying 95pc chance that effect is within log2FC of 20.
% > qnorm(0.975, 0, 10)
% [1] 19.59964

% Two examples to demonstrate: the top 2 in RMA meta-analysis.
% (see section of code at end of 6.0\_de\_meta.R script)

\section{Results}

\subsection{Comparison to Sobolev R vs. NR}

Stuff from 1st year report.

\subsection{modules}
% https://www.jacionline.org/article/S0091-6749(17)31766-9/fulltext#sec2.4
The reduced efficacy of vaccination has also been linked to excessive inflammation for influenza,31 yellow fever,32 tuberculosis,33 and hepatitis B34 vaccines.
