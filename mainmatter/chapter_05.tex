%
% Chapter 5
%

\chapter{multiPANTS}

\section{Introduction}

Why do some people not respond?

Explore time series transcriptomic
Find out optimal spline degree.

Creating composite features to conduct genetic associations on.

Identifying signatures of response.

\section{Methods}

immunomods
% > pheno %>% filter(RNAseq_Sample_Status == 'Pass') %>% distinct(PANTS.ID, .keep_all=T) %>% select(Primary_Response, Ever_Immunomodulator) %>% table
%                 Ever_Immunomodulator
% Primary_Response  NO YES
%            FALSE  24 129
%            TRUE   28 159
% > pheno %>% filter(RNAseq_Sample_Status == 'Pass') %>% distinct(PANTS.ID, .keep_all=T) %>% select(Primary_Response, On_Immunomodulator_At_Baseline) %>% table
%                 On_Immunomodulator_At_Baseline
% Primary_Response FALSE TRUE
%            FALSE    85   68
%            TRUE     83  104
% > pheno %>% filter(RNAseq_Sample_Status == 'Pass') %>% distinct(PANTS.ID, .keep_all=T) %>% select(Primary_Response, On_Steroids_At_Baseline) %>% table
%                 On_Steroids_At_Baseline
% Primary_Response FALSE TRUE
%            FALSE    83   70
%            TRUE    133   54

In the IFX+ADA cohort, 
    DE
    PR vs PNR baseline
    PR vs PNR and w14

n patients with data for each number of visits

\subsection{Covariates to use}

Sex
Age
BMI
Age of Onset
Crohn’s Surgery
Ever Immunomodulator
Current Smoker
PCA
Proportions of the 6 cell types: CD4+ T cells, CD8+ T cells, B cells, NK cells, monocytes, and granulocytes

\section{Results}

\section{Discussion}

