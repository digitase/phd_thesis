%
% Chapter 5
% Discussion
%

\chapter{Discussion}

\begin{outline}

summary of chapters

    baseline prediction might require much larger sample sizes

    Tie ch 2 to 3 using baseline predictors?
    A response eqtl is not always a response eqtl

    Limitations, and the perfect study.
        recommendations for design of future studies

    suffer from core set of limitations 

What is response?
why is it so difficult to make reliable response prediction, esp at baseline?

    need very large 
        replicability: modules more robust

        % caution on reports about any single gene e.g. SIGLEC10 ch4

    improve efficiency
        move away from dichotomania
        % https://discourse.datamethods.org/t/responder-analysis-loser-x-4/1262
        % Responder analysis is a 4-fold loser. It fails to give the needed clinical
        % interpretation, has poor statistical properties, increases the cost of studies
        % because binary outcomes require significantly higher sample sizes, and raises
        % ethical issues because more patients are required to be randomized than would
            in ch 3, used TRI
                also need the cellular response, beyond ab titires cao2016SystemsImmunologyAntibody
            in ch 4, 
                a very complex phenotype
                May be more powerful to directly with the constituent phenotypes directly (CRP level, HBI score, whether there was escalation in steroid treatment, or exit for treatment failure)

                predictive claims

                also, similar to beyond Abs
                response has many definitions: subphenotypes that are both positive and negative may be complex traits with different architectures: immunogenicity, primary response, loss of response, remission rate, adverse fx

    or response itself is the problem?
        rely of stability of 'responder', something intrinsic
            esp in DGE models (more below)
        "60\% of the time, it works every time"
            \url{https://discourse.datamethods.org/t/responder-analysis-loser-x-4/1262}
            senn2016MasteringVariationVariance
            senn2018StatisticalPitfallsPersonalized

            reported failure rates (single-measure) do not necessarily reflect that there is something inherent to an individual that causes it \url{https://www.ncbi.nlm.nih.gov/pmc/articles/PMC524113/} senn2016MasteringVariationVariance

                in ch4: repeated dosing helps guard against this, some hope that NR is stable
                and n of 1 trials
            this is where longitudinal comes in

The design of more longitudinal cohorts in the future

    sample size is the main problem
        complex traits and disease already moved out of candidate gene era
        vaccine and drug response traits lagging due to sample size requirements

        prespecify interactions of interest

    need a strong causal anchor: this is going backwards towards gwas, which requires N

    more n
        just because n sufficient for eqtl at 1 condition
        underpowered when looking for interactions (implied for reQTL, even more problematic for e.g. further cell count interatcions)
        similar to stratified analysis problem in clinical trials

    ofc if avoid bulk, no prob with cell counts:
        Even if cell type interaction/proxy gene approach
        Cannot distinguish between correlated cell types

    more chance for in vivo
        but how to take advantage of it

    systems immunology/vacc still needed: generate mol phenotypes, in the right context,
    but make sure to include genotypes
    Moving out of correlation land

    all the change score nonsense

    gone are the days of GWAS marginals (just a screening approach)
        under time pressure, or convention
        throwing in covariates

    as     complex disease genetics moves 
    computation not limits correct stat

what data to get from these trials?
reQTL on bulk in vivo has huge challenges
more n won't help

    composition explains a huge amount of var in bulk farahbod2020UntanglingEffectsCellular

    can correct, but see 30 for age correction
        ch3, coarse
        ch4, even coarser

        rarer pops, so can always do more
        is it ever enough?

    some are implicitly accounted for in peer
        but interactions with geno are not guaranteed to be

    effect not absent in vitro
        technology itself is proportional, with norm based on ng input, then random sampling during sequencing

    but effect is certainly EXARCERBATED by in vivo
        heterogeneity is confounded with recruitment!!!
        not just sample var, but active recuitment changes in cell prop e.g. recruitment
            why should this be considered a con of in vivo?

    need to recognise 
        bulk and sorted eQTL differ in interp
            and correction using cell prop covariates does not unify them

        also, it's not the same as intervertion to set cell counts e.g. FACS

            control is not intervention
                % http://bayes.cs.ucla.edu/PRIMER/ch3-preview.pdf
                The difference between intervening on a variable and conditioning on that
                variable should,hopefully, be obvious. When we intervene on a variable in a
                model, we fix its value. Wechangethe system, and the values of other variables
                often change as a result. When we condition on avariable, we change nothing; we
                merely narrow our focus to the subset of cases in which thevariable takes the
                value we are interested in. What changes, then, is our perception about
                theworld, not the world itself.

    unclear what benefit would outweigh this
        imo, needs consideration for future study design

    must no longer be divorced from rigourous statistical theory

        statistical considerations from small scale studies
            Two disciplines of statistical genomics : Neither can be neglected

            in chapter 3, i discussed in brief
                change is not ancova
                stratified is not interaction
                yet many studies out there the that use one or the other to answer the same biological reQTL questions

        beware the effects of normalisation (on top of stratification)

        beware the effects of model misspec
            missing interaction terms

        esp when considering biologically meaningful replication of reQTL
            all above must line up

as more n, allows
Finer and finer context
    in the intro: gwas to function pipeline
    now, the future

    More timepoints
    Allele-specific expression changes dynamically during T cell activation in HLA and other autoimmune loci
    % https://www.nature.com/articles/s41588-020-0579-4

    More conditions
    e.g. 250 condition ASE % https://www.ncbi.nlm.nih.gov/pmc/articles/PMC5131815/
    e.g. StructLMM 
        Identifies eQTLs with GxE, where the number of environments in E is large (modelled as a random effect)

        vqtls: e.g. % doi: https://doi.org/10.1101/2020.07.28.225730 

    as datasets and conditions get larger, proportion of eGenes is going to be 100pc, then the question is what are the most relevant ones
    % We observed that cis-eQTLs can be detected for 88% of the studied genes in vosa2018UnravelingPolygenicArchitecture
    % In contrast, trans-eQTLs (detected for 37% of 10,317 studied trait-associated variants) were more informative.

    Era of single cell. (ultamite context)
        refined over modules
        
        1st
        Single-cell RNA sequencing identifies celltype-specific cis-eQTLs and co-expression QTLs
        https://www.nature.com/articles/s41588-018-0089-9

        "Single-cell eQTLGen Consortium: a personalized understanding of disease"
        https://arxiv.org/abs/1909.12550

        Optimal design of single-cell RNA sequencing experiments for cell-type-specific eQTL analysis
        https://www.biorxiv.org/content/biorxiv/early/2019/09/12/766972.full.pdf

        Single-cell genomic approaches for developing the next generation of immunotherapies Ido Yofe, Rony Dahan and Ido Amit

        % Lähnemann, D., Köster, J., Szczurek, E., McCarthy, D. J., Hicks, S. C.,
        % Robinson, M. D., Vallejos, C. A., Campbell, K. R., Beerenwinkel, N., Mahfouz,
        % A., Pinello, L., Skums, P., Stamatakis, A., Attolini, C. S.-O., Aparicio, S.,
        % Baaijens, J., Balvert, M., Barbanson, B. de, Cappuccio, A., … Schönhuth, A.
        % (2020). Eleven grand challenges in single-cell data science. Genome Biology,
        % 21(1). https://doi.org/10.1186/s13059-020-1926-6

        reQTL detection: bulk, sorted, sc
        current sc will only detect highly expressed genes

        reqtl followup using single cell cell type specific expression
        % https://journals.plos.org/plospathogens/article?id=10.1371/journal.ppat.1008408

        devries2020IntegratingGWASBulk
        Subsequently, scRNA-seq data was used to pinpoint the potential cell type in which the response QTL effects manifest themselves
            but really they only looked at cell type specific expression, not sc eqtl mapping

and allows
more phenotypes
        
        disease specific biobanks e.g. ibd bioresource/predicct

    % Our GWASfor 54 functionally relevantphenotypes oftheadaptiveimmune system in 489 healthy individuals identifieseight genome-wide significant associations explain-ing 6%–20% of variance.
    % https://www.cell.com/cell-reports/pdf/S2211-1247(18)31493-1.pdf

    PheWAS\autocite{verma2017CurrentScopeChallenges}
    % https://onlinelibrary.wiley.com/doi/full/10.1111/imm.12195
    PheWAS has the advantage of identifying genetic variants with pleiotropic properties.

BUT basically, Extending the sample size is key

    longitudinal studies are smallish

    but follow very similar trends to complex disease and traits
        candidate genes identified in small studies
        GWAS

    be careful not to slice and dice

    take advantage of large resources
    e.g. biobank
    e.g. 1000IBD
    e.g. IBD bioresource
        TWAS: predicted gene expression, then associate with phenotypes

finally translation and prediction
    a diff question althogether

    Prediction is not inference (the rift between philosophies, see 2 x 2 schools papers)
        % Calculating the sample size required for developing a clinical prediction model
        two chapters of mine are merely descriptive, only form the foundation for next leap should be to causal or predictive
            have touched on suggesting to build a predictive model
        but have always treated the R as indep in DGE
        kinda strange: response is a organism pheno should be downstream of E

            Is response , the main predictor
            A property measured at w14
            Is actually A time invariant property of the individual?

            for modelling convenicnece
            cant make causal claims anyway
                without a control, we cannot observe the counterfactual of what if an individual was a non-responder

            correlates of protection

            no error
                assuming otherwise would stray into the realm of error in variables models
            
                we are effectively assuming that R is an intrinsic thing, and is determined without error

        gene signatures, rise and fall
            expression as a biomarker

            success stories \url{https://www.nature.com/articles/415530a} 
            and
            rise and fall paper: 15 years experience from cancer: list challenges to clinical implementation from Discussion

            % https://www.annalsofoncology.org/article/S0923-7534(19)36531-7/fulltext
            Table 1Evidence-based criteria for a prognostic gene signature in the path from the laboratory to clinical practice

        needed to reach the status of signature
        
    efron2020PredictionEstimationAttribution prediction is inherently different
        % For those of uswho have struggled to find “significant” genes in
        % a microarray study,8 the almost perfect prostate cancer predictions
        % of randomforests and gbmhave to come as a disconcerting
        % surprise.
            not clear DGE -> prediction is necessary
            prediction may still be possible even with little significant single gene diffs

            is the gene by gene approach good for pred?
        
        need to repeat DGE within folds
            no guarantee that top DGE will be top predictors

        need to convert the language e.g. logFC, to prediction e.g. CV error

        cell props not ideal for prediction?
            gaujoux2019CellcentredMetaanalysisReveals: "These results suggest that the predictive power of reported gene signatures is largely based on cell subset proportion differences, whose increase in colon biopsies of non-responders may serve for predictive purposes."

        % beyond dge
        % TODO: https://bmcgenomics.biomedcentral.com/articles/10.1186/1471-2164-13-356

        potential methods

            Sparse partial least squares regression for simultaneous dimension reduction and variable selection
            % https://www.ncbi.nlm.nih.gov/pmc/articles/PMC2810828/

             Sparse Partial Least Squares Classification for High Dimensional Data 
            % https://pubmed.ncbi.nlm.nih.gov/20361856/

        sysvacc: need genetics to move beyond prediction

    might even be said that whole blood biological interps so limited that prediction may be the only feasible goal

    and to Cost-effectiveness and clinical implementation
        if you can identify NRs, what are you going to do about it?

        what does it take for a gene expression pred model to be clinically useful?
            take lessons from cancer

the missing link needed for causality

    need causal models

        avoid "low input, high throughput, no output"
        without causality no talking about 'mechanism'

    need statistical integration
        problems with connection between dge and reqtl, as in prev ch

        simple overlap will not work

            e.g. other papers that talk about 'integrating' recognise this
            % franco,
            % https://journals.plos.org/plosone/article?id=10.1371/journal.pone.0180824
            % devries2020IntegratingGWASBulk

    option:
        Causal Inference Test for a Binary Outcome

        % TODO
        % https://onlinelibrary.wiley.com/doi/full/10.1002/gepi.22061
        % The CIT is specifically designed to test whether a variablemediates the association between (and is the only causal linkbetween) a genetic locus and a quantitative trait. It is moreflexible than MR because it does not assume that the geneticvariant is chosen specifically to be an instrument for the medi-ator. Due to the way the test is constructed, the CIT is alsoimmune to problems of pleiotropy and reverse confounding.
            not tried in ch2 due to low power, power not evaluated for ch4

            Causal Inference Methods to Integrate Omics and Complex Traits
            \url{http://m.perspectivesinmedicine.cshlp.org/content/early/2020/08/17/cshperspect.a040493.abstract?s=09}

            % TODO: mediation can exist in the absence of a 'significant' total effect G -> pheno
            % http://davidakenny.net/cm/mediate.htm
            % In the opinion of most though not all analysts, Step 1 is not required. (See the Power section below why the test of c can be low power, even if paths a and b are non-trivial.)

    option: MR with expression 
        instrument variables are really just perfect mediators

        genetic instruments
        as demonstrated by e.g. qinqin, use (r)eqtl as genetic instruments

        e.g. sigeclec10
        we have eqtls for it, but a bad idea since indep

        does this approach require external datasets?

        MRLocus: identifying causal genes mediating a trait through Bayesian estimation of allelic heterogeneity
        \url{https://www.biorxiv.org/content/10.1101/2020.08.14.250720v2}

unification
    triangulation
    immunology and vaccine dev: deep phenotyping, small cohorts achieved -> larger cohorts
    human genetics and gwas: large cohorts achieved -> deeper phenotyping

    more intermediates:
        The proteome: disconnect between regulatory layers
        % https://pubmed.ncbi.nlm.nih.gov/32709985/
        % https://www.nature.com/articles/s41576-020-0258-4
        % Although protein and mRNA levels typically show reasonable correlation, we describe how transcriptomics and proteomics provide useful non-redundant readouts.

        % e.g. anti-TNF does not imply reduction in TNF expression, but we see only downstream fx through regulation

    MOFA
    multiomics
        % https://twitter.com/Eric_Fauman/status/1266128178692718598?s=19
        Pqtls more accurate?

    already lots of expression data

    combining systems immunology studies of genetic arch of immune parameters
            e.g. dejager2015ImmVarProjectInsights, zalocusky201810000Immunomes
        giving gmore intermediate phenotypes
        layering of evidence (triangulation)
            molqtls (e.g. foreshadowing, pqtls), MR, CIT, coloc

    with GWAS of immune phenotypes

% \1 Why care?
%
%     \2 polygenic scores, prs: marker for diagnosis
%
    % first
    % 13. Wray, N.R., Goddard, M.E., and Visscher, P.M. (2007). Prediction
    % of individual genetic risk to disease from genome-wide
    % association studies. Genome Res. 17, 1520–1528.
%
%         \3 use in the clinic
            % doi: http://dx.doi.org/10.1101/19013086
%             \4 e.g. polygenic background can modify penetrance
%
%         \3 but challenges from:
%
%             \4 ancestry effects
            % https://twitter.com/skathire/status/1202554709107712001?s=09
%             \4 need expanding into global populations, global biobanks e.g. Gains from Africa H3Africa, japanese biobanks
%
            % Variable prediction accuracy of polygenic scores within an ancestry group
            % https://elifesciences.org/articles/48376
%             \4 non-ancestry effects
%
%     \2 pathway analysis: "the great hairball gambit"
%
%     \2 pathway prs
%         \3 challenge is variant to gene assignment/mapping
%             \4 e.g. restrictions to fine mapped eQTLs
%
%     \2 Understand mech. of causal genes: molecular pathogenesis
    % word mechanism thrown around
    % Mechanism is important because of possibility for intervention. Here, causal is important

% \subsection{From target gene to candidate drug}
%
% https://journals.plos.org/plosgenetics/article?id=10.1371/journal.pgen.1008489
% Are drug targets with genetic support twice as likely to be approved? Revised
% estimates of the impact of genetic support for drug mechanisms on the
% probability of drug approval
    % \2 how to drug a complex disease with no single 'candidate gene'?
% GWASing and fine-mapping complex diseases like IBD turns out a large number of common causal variants with small-effect sizes.
% - Is polygenicity a population or individual property? i.e. are most individual IBD cases driven solely by a distribution of small-effects, or do most patients also have 1 or more large-effect rare variants that point out priority targets for their own personalised treatment?
% - Do many of these common causal variants e.g. converge to hit on the same pathways?
% - Otherwise, what is the use of these target discovery pipelines that output ranked lists of target genes? Could a drug designed to modulate a single protein target be expected to work for a large number of patients?
    %
    % e.g. schizo is usually polygenic, future drug development could benefit from taking a multi-target approach \autocite{visscher201710YearsGWAS}
    %
    %     \3 e.g. of successful GWAS -> drug target
    %         \4 drug targets with genetic support are more likely
    %
    %     \3 building allelic series

\end{outline}

