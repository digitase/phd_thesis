%
% Chapter 5
% Discussion
%

\chapter{Discussion}
\label{ch:discussion}

% TODO: check each section provides some solutions
\begin{outline}

\gls{HIRD}
\gls{PANTS}
\1 In this thesis, ... <a summary of the thesis projects>
    \2 Chapters 2 and 4 focus on describing the transcriptomic response of circulating immune cells to perturbation with uncontrolled longitudinal designs
        \3 looking at how expression changes over time, and differences in transcriptomic response associated with phenotypic definitions of response
        \3 associations are identified at a gene and gene set level
    \2 Chapter 3 (and the last part of chapter 4) focus on finding common genetic variants associated with transcriptomic response to perturbation
        \3 reQTLs are identified at lead eQTL variants for each gene
        \3 this is a first foray into defining causal molecular mechanisms that explain individual variation in response to these perturbations
    \2 Each chapter discusses the results and limitations of each study, but being very similar studies in design and analysis, they also suffer from a core set of limitations.
    \2 This chapter discusses these, and considerations for design and analysis of similar longitudinal studies in the future to better understand the biology of immune response to drugs and vaccines.

\section{Detecting robust associations between expression and phenotype}

In both \cref{ch:hird_DGE,ch:hird_reQTL}, I focused on identifying genes with robust differential expression over time after immune perturbation, 
or expression associated with phenotypic response variables: antibody titres or clinical anti-\gls{TNF} response.
The perturbations had strong effects on large proportions of the blood immune transcriptome, 
often resulting in thousands of highly significant associations when comparing pre- and post- perturbation timepoints.
In comparison, it was much more challenging to identify robust single gene associations with response phenotypes.
In \cref{ch:hird_DGE}, associations of day 7 expression with antibody response from \textcite{sobolev2016AdjuvantedInfluenzaH1N1Vaccination} were replicated in my analysis of the array data, but not in \gls{RNAseq}, or in a meta-analysis.
In \cref{ch:multiPANTS}, baseline associations with anti-\gls{TNF} from the literature,
including at \gene{TREM1}, previously reported by two independent groups \autocite{gaujoux2019CellcentredMetaanalysisReveals,verstockt2019LowTREM1Expression}, were not replicated in my analysis of the \gls{PANTS} cohort.
The biological effect size of expression on response is likely to be small, eclipsed by other sources of variation: 
measurement platform, response definitions, sample characteristics, and noise.
Power calculations for \gls{DGE} are non-trivial,
and it is often not known what a reasonable effect size to assume might be for exploratory studies.
Many experiments choose parameters like sample size and sequencing depth based on rules of thumb \autocite{conesa2016SurveyBestPractices}, or to be comparable to existing ones in the field.
The idealistic suggestion is to increase sample size, 
but resource and ethical constraints mean samples are often just the largest that can be feasibly obtained.
Rather than creating new cohorts, a logistically-efficient strategy is is collecting samples from individuals enrolled in drug and vaccine trials,
but care must be taken to ensure the trial is powered both for its primary endpoints,
and for planned transcriptomic analyses.
% NOTE:
% https://stats.stackexchange.com/questions/176384/do-underpowered-studies-have-increased-likelihood-of-false-positives
% The basic problem with underpowered studies is that, although the rate of false positives (type I error) in hypothesis tests is fixed, the rate of true positives (power) goes down. Hence, a positive (= significant) result is less likely to be a true positive in an underpowered study. This idea is expressed in the false discovery rate [2], see also [3]. This seems what the quote refers to.
% An additional issue often named regarding underpowered studies is that they lead to overestimated effect sizes. The reasons is that a) with lower power, your estimates of the true effects will become more variable (stochastic) around their true value, and b) only the strongest of those effects will pass the significance filter when the power is low. One should add though that this is a reporting problem that could easily be fixed by discussing and reporting all and not only significant effects.
In cases where high power is not guaranteed and small effects are possible,
when reporting and interpreting associations that have been determined to be significant based on some threshold, 
one should always be cognizant of winner's curse \autocite{huang2018PowerFalseDiscovery}.
% This is doubly the case when effect sizes are known to be small,
% as years of failed replications from underpowered candidate gene studies of complex traits have demonstrated \autocite{border2019NoSupportHistorical}.
% Likely plays a part in why early systems vaccinology studies didn't include host genetics.

Another consideration is how best to distribute a fixed sample size between depth (number of individuals) and richness (number of timepoints, phenotypes, data types).
For a phenotype as dynamic as immune response, longitudinal sampling is required.
In \cref{ch:hird_DGE}, sampling demonstrated a distinct jump from day 1 innate to day 7 adaptive immune expression profiles,
but the kinetics of the transition are not clear. 
In hindsight, responses could have peaked earlier or later in different individuals,
and variation in the speed of response can not be examined without denser sampling.
In \cref{ch:multiPANTS}, expression differences between responders and non-responders were apparent from week 14, but it is not known if differences actually appear much earlier.
Future analysis of a (small) number of available samples from day 3 after initiating anti-\gls{TNF} treatment may uncover associations in the early innate immune response.

Rich sampling also offers analytical advantages.
Having repeated measures from the same individuals allowed modelling of within-individual covariance in \cref{ch:hird_DGE,ch:multiPANTS}, improving statistical efficiency.
The spline model in \cref{ch:multiPANTS} enabled separation of responders and non-responders based on expression trajectory over multiple timepoints.
However, in each case models only incorporated two data types: expression and phenotypic response.
Studies in the systems vaccinology field have demonstrated how integrating networks of multiple data types identifies correlates and predictors of response not only in the transcriptome, 
but in multiple layers of the immune system \autocite{li2017MetabolicPhenotypesResponse}.
In \gls{HIRD}, longitudinal \gls{FACS} and cytokine measurements are available for this form of integrative modelling.

% \1 Even if the raw number of samples goes up, subgrouping quickly attenuates the effective sample size.
%     \2 e.g. looking for timepoint, responder, cell type interactions
%     \2 pre-specify interactions of interest at the power analysis stage in design of new experiments
%     \2 for existing datasets, a workaround like the two stage strategy in ch3, assuming that interactions are only interesting at significant main effects

Fortunately, systems immunology studies quantify on a global scale.
Genes in the immune system are not independent; just as variation increases uncertainty, covariation reduces it\footnote{Exploratory Data Analysis. R for Data Science. \url{https://r4ds.had.co.nz/exploratory-data-analysis.html}}.
In \cref{ch:hird_DGE}, estimates from multiple genes were used to build an informative empirical prior for between-platform heterogeneity.
Throughout the thesis, I make extensive use of enrichment analyses with gene sets defined by prior biological knowledge,
to detect subtle but coordinated changes based on the expression of multiple genes.
General purpose gene sets may be less relevant in immune cells, 
so I used \glspl{BTM} \autocite{chaussabel2008ModularAnalysisFramework,li2013MolecularSignaturesAntibody} tailored for immune gene expression in blood.
Many significant associations with vaccine antibody response and clinical drug response were identified in \cref{ch:hird_DGE,ch:multiPANTS},
and my expectation is that these should be more replicable than any single gene associations I reported (e.g. \gene{SIGLEC10} from \cref{ch:multiPANTS}).
While the effect size of a single gene may vary from sample to sample due to noise, 
a summary measure computed from multiple genes should be more robust.
Indeed, some module associations between baseline expression and antibody response found in \cref{ch:hird_DGE} were reported in previous studies of seasonal influenza vaccines.
Most systems vaccinology studies aiming to identify consistent associations with vaccine response over multiple cohorts and sampling years,
focus their analyses at the gene set level \autocite{tsang2020ImprovingVaccineinducedImmunity}.
Gene set analyses cannot, however, be divorced from examining the genes within them,
as the genes that drive set associations can differ between apparent replications,
and the mapping between genes and sets is one-to-many.

\section{Responder analysis}

A key determinant of how well the models in this thesis might correspond to reality lies in the assumed model for phenotypic response.
This has been extensively discussed in the context of randomised controlled trials, but similar issues pertain to response definitions in observational studies \autocite{senn2018StatisticalPitfallsPersonalized}.
Assume a hypothetical drug or vaccine where 60\% of a sample of individuals have observed: \enquote{60\% of the time, it works every time}\footnote{Apatow, J., McKay, A. \& Ferrell, W. Anchorman: The Legend of Ron Burgundy (2004).}.
This is compatible with a stable 60\% success rate in 100\% of individuals (variation in observed response is entirely due to chance);
or a stable 100\% success rate in 60\% of individuals and 0\% in the other 40\% (response is highly personal)---most likely the truth is somewhere in between.
In the first scenario, it is difficult to imagine identifying robust baseline associations with response.

One needs to establish how correlated phenotypic response is over time within the same individual,
\todo{I mean phenotypic response here, not (biological or technical) variation in expression measurements}
and to compute within-individual variation requires replication at the level of the individual.
% \2 also applies to continuous measures of response, as what is being considered is that it only makes sense to talk about response if the response measure is tightly correlated within an individual over time
The same individual must be \emph{perturbed and measured} more than once \autocite{senn2016MasteringVariationVariance}.
This is not always possible in practice.
In \cref{ch:hird_DGE}, antibody response was defined based on a single measurement after a single dose.
but measuring response after a hypothetical second dose would quantify a different phenotype: 
the secondary immune response based on vaccine-induced immune memory.
In \cref{ch:multiPANTS}, all patients received repeated doses interspersed with sampling timepoints,
and the expression differences between clinical responders and non-responders seen at week 14 (just after clinical response was assessed) were maintained at week 30 and week 54,
suggesting the initial designation of non-responders is not due to chance, but some characteristic of those patients' disease states.
%
% \2 note for the specific case of genetic factors, the field of complex trait genetics has long had a solution for this: twin studies are analogous to measuring the same individual twice
%     \3 For phenotypes like vaccine response in ch2 that can be, twin studies have already demonstrated heritability of certain response phenos like influenza vaccine Ab titres \autocite{brodin2015VariationHumanImmune}
%     \3 In pharmocogenetics, twin studies are less common \url{https://www.sciencedirect.com/science/article/abs/pii/S0928098719300326} and have not been done for response to anti-TNFs (the ch4 setting)
%     \3 repeated crossover trials can be a feasible alternative to twin studies \url{https://www.researchgate.net/profile/Laszlo_Endrenyi/publication/13553486_Hypothesis_Comparisons_of_inter-_and_intra-individual_variations_can_substitute_for_twin_studies_in_drug_research/links/59e97c2e458515c36370e6a7/Hypothesis-Comparisons-of-inter-and-intra-individual-variations-can-substitute-for-twin-studies-in-drug-research.pdf} \autocite{senn2016MasteringVariationVariance}

Even if response is actually a stable personal characteristic, one still needs to select an appropriate mathematical definition.
As discussed in \cref{subsec:hird_dge_TRI}, a binary definition of response based on dichotomisation is inefficient and biologically implausible.
I instead used the \gls{TRI} in \cref{ch:hird_DGE,ch:hird_reQTL}, a continuous change score combining \gls{HAI} and \gls{MN} titres, residualised on the baseline titres.
In \cref{ch:multiPANTS}, the binary clinical response phenotype is based on a complex decision tree with many inputs.
Defining dichotomies based on multiple inputs can lead to discontinuities and non-monotonicity in response probabilities under small changes in inputs \autocite{senn2005DichotomaniaObsessiveCompulsive}.
Neither approach is ideal, but pragmatism did come into play.
For \gls{DGE}, the simplest models encode phenotypic response variable as a single independent variable, with expression as the sole dependent variable.
Both \gls{TRI} and the \gls{PANTS} clinical response definition provided that single independent variable.
In hindsight, variation in response definitions likely contributes to difficulties in replicating associations between studies,
so it may be more sensible to model on the components phenotypes themselves (e.g. log \gls{HAI} and \gls{MN} titres, \gls{CRP} levels and \gls{HBI} scores).

% \1 bias in non-randomised studies
    % \2 In RCTs, the treatment effect would be defined against the effect in the control group
    % \2 In prospective cohort studies, the treatment effect is determined between two subgroups within the cohort

\section{Challenges in the interpretation of bulk expression data}

\1 bulk data is a mixture of cell types with different expression patterns
    \2 one of the largest sources of variation in bulk expression data is cell proportions 
    \2 the more cell type-specific a gene's expression, the more it is affected by cell proportions \autocite{farahbod2020UntanglingEffectsCellular}
        \3 perfectly cell type-specific genes are often called marker genes, and are used in deconvolution methods to estimate cell proportions from bulk data when they are not directly measured
    \2 both variation from in true variation in proportions, and sampling noise
    \2 in this thesis, estimated by:
        \3 In HIRD ch 3, xcell deconvolution from expression data
        \3 In PANTS ch 4, deconvolution from methylation data
    \2 these are fit as covariates in models
        \3 could act as precision variables (for sampling noise in non cell type-specific genes),
        \3 but also mediators for cell type-specific genes, where the interpretation of DGE effects changes (perturbation -> cell prop -> expression)
    \2 but coarse grained (just the main cell types) adjustment is not always sufficient, you'll still find associations that are proxies for rare cell types \url{https://www.biorxiv.org/content/10.1101/2020.05.28.120600v1}
        \3 Recommend the use of e.g. PEER, as in ch3/4 eQTL analysis, some of those factors are likely to be rarer cell types

% TODO: read Where Are the Disease-Associated eQTLs?
% https://www.cell.com/trends/genetics/fulltext/S0168-9525(20)30209-2

\1 <eQTL analysis>
    \2 analogous to DGE where independent variable of interest is now genotype
    \2 as discussed in \cref{subsec:hird_reQTL_methods_cellTypeInteraction}, it is model misspecification to not have the genotype x cell proportion interaction if the effect of genotype changes depending on cell proportions
        \3 in fact, it is popular to use genotype x cell proportion (or a proxy of) interactions to call cell type-specific eQTLs e.g. \autocite{westra2015CellSpecificEQTL,kim-hellmuth2020CellTypeSpecific}
        \3 but you cannot distinguish between correlated cell types this way

\1 <in vivo reQTL analysis>
    \2 here the cell props are causally affected by the perturbation (active recruitment, differentiation, proliferation of immune cells)
    \todo{would an in vitro stimulation of a mixture of cells be immune to this? differentiation and proliferation could still occur}
    \2 consider the contrived case, where vaccine perturbation causes active increase of a rare cell type that is near absent at baseline, but is a greatly increased proportion of the bulk mixture after perturbation.
    \2 further assume this is true for all individuals (no variation in cell proportions).
    \2 an eQTL specific to this cell type will show up as a reQTL in the post-perturbation timepoint because expression of that cell type contributes more to the bulk mixture.
        \3 this will happen even if you adjust for proportion of the cell type, since there is no variation between individuals, it will only offset the regression line
        \3 this does not require the gene to be upregulated on average, as the effect of interest is G -> delta E, not delta E.
    \2 if the aim of a reQTL study is to identify variants that causally affect transcriptomic response to perturbation, 
        such that if we changed the genotype of an individual (observe the counterfactual), the transcriptomic response would be different.
        \todo{assumes eQTL is finemapped}
    \2 does this achieve that aim? 
        yes, as the change in expression after perturbation differs when you change someone's genotype,
        as there is no difference between genotypes at baseline,
        but there is after perturbation.
        \todo{I cannot figure out what sort of conditioning would need to be done to block this sort of effect even if you had perfect cell proportion measurements}
        % NOTE: protected from vaccine -> E and genotype -> E reverse causality, but still...
    \2 for genotype to explain more variance in bulk expression after perturbation, there are multiple compatible mechanisms
        \3 a cell type with a cell type-specific eQTL increases in proportion (recruitment) 
        \3 a gene with an eQTL increases in expression (activation or recruitment depending on if you condition by cell proportions)
        \3 the effect of a cell type-specific eQTL increases within that cell type (activation)
    \2 one could imagine crude ways to falsify each hypothetical model e.g. interaction/coloc with each cell type (ch3), but always at risk of missing a cell type
    \2 the challenge is distinguishing these scenarios in bulk in vivo data to form mechanistic hypotheses for expression response
    % \2 Not yet clear whether \textit{in vivo} reQTL have any utility on top of \textit{in vitro} reQTL for interpreting GWAS loci: not that many studies, and complex interpretations.
    % Not clear how in vivo reqtls help gwas priitisation
    % May identify extra signals
    % But they themselves are harder to interpret, often requiring in vivo eqtls

\1 future studies should consider alternatives to bulk before they can fully take advantage of benefits of in vivo design
    % http://bayes.cs.ucla.edu/PRIMER/ch3-preview.pdf
    % The difference between intervening on a variable and conditioning on that variable should,hopefully, be obvious.
    % When we intervene on a variable in a model, we fix its value.
    % Wechangethe system, and the values of other variables often change as a result.
    % When we condition on avariable, we change nothing; we merely narrow our focus to the subset of cases in which thevariable takes the value we are interested in.
    % What changes, then, is our perception about theworld, not the world itself.
    % \2 control is not intervention
    \2 rather than conditioning on cell proportion, effectively looking at the subset of data with same cell proportions, just control them
    \2 If you want a mechanistic interpretation, need to control
    \2 single-cell RNAseq has the advantage of getting expression and composition simultaneously, and can be comparable in cost to bulk on FACS sorted cells \url{https://arxiv.org/ftp/arxiv/papers/1909/1909.12550.pdf}
        \3 but lower coverage of the transcriptome due to drop out, and smaller sample sizes
    \2 do DGE/map eQTLs within each cell type 
        \3 combines advantage of in vivo stimulation with the controlled cell composition of in vitro studies
    \2 can also be used to annotate bulk reQTLs such as those found in ch3 e.g. find the likely cell types for bulk reQTL effects by looking at eGene expression in single cell data \autocite{devries2020IntegratingGWASBulk}

% more n allows more phenotypes
%
%         disease specific biobanks e.g. ibd bioresource/predicct
%
%    Our GWASfor 54 functionally relevantphenotypes oftheadaptiveimmune system in 489 healthy individuals identifieseight genome-wide significant associations explain-ing 6%–20% of variance.
%         https://www.cell.com/cell-reports/pdf/S2211-1247(18)31493-1.pdf
%
%     PheWAS\autocite{verma2017CurrentScopeChallenges}
%         https://onlinelibrary.wiley.com/doi/full/10.1111/imm.12195
%         PheWAS has the advantage of identifying genetic variants with pleiotropic properties.
%
% and Many more conditions
    % e.g. 250 condition ASE % https://www.ncbi.nlm.nih.gov/pmc/articles/PMC5131815/
    % e.g. StructLMM
    %     Identifies eQTLs with GxE, where the number of environments in E is large (modelled as a random effect)

\section{From association to prediction}

\1 I have now discussed both the independent and dependent variables used in descriptive models of expression vs response
    \2 so far, always put expression as dependent, and response status as independent, and estimated the effect size of response status on expression
    \2 did this even when response is measured after expression (baseline expression)
        \3 this is the assumption that response is a stable individual characteristic
        \3 also assumes it is measured without error, otherwise would stray into the realm of error in variables models
    \2 but might also want to consider if given expression, you could predict response (probability) as a dependent variable
    \2 this is also directly relevant to clinical goal of predicting patient response from baseline expression
    % bulk data doesn't matter
    
\1 a large p >> n prediction problem
    \2 given the few gene-level associations with response is it futile to try?
    \2 \textcite{efron2020PredictionEstimationAttribution} provides an encouraging counterexample on n=102 (52 prostate cancer, 50 controls), p=6033 genes
        \3 random forest to predict cancer status from gene expression: 2\% test set error
        \3 removed the 348 genes with positive importance scores, these are all the genes that were ever involved in a splitting rule
        \3 rerun on resulting 102 x 5685 matrix, still a similar error rate
        \3 repeat that process, removing 364 genes. rerun on resulting 102 x 5321 matrix, still a similar error rate
    \2 performance of such purely predictive models is dominated by combining many weak predictors
    \2 pure prediction is an easier problem than the traditional approach trying to attribute statistical significance to a few strong predictors
    \2 this may be why there is sometimes little overlap in the gene signatures from different studies on the same trait

% What sample size?
    % https://www.bmj.com/content/368/bmj.m441
    % Calculating the sample size required for developing a clinical prediction model

\1 Need to consider what it takes for a gene expression prediction model to be clinically useful?
    \2 Useful to look at examples from cancer, one of the earliest fields where the approach generated interest
    \2 Despite the early gene expression signatures being reported in the early 2000s, only a handful in use \autocite{chibon2013CancerGeneExpression} \url{https://clincancerres.aacrjournals.org/content/21/21/4743}
    \2 many hurdles: need to be accurate enough to base changes in therapy on, but incrementally more accurate than existing clinical markers, and cost effective. \url{https://doi.org/10.1093/annonc/mdw307}
    \2 measuring a whole transcriptome is likely not cost effective. what are potential methods that might be?
    \2 for predicting vaccine response from transcriptomic data, popular methods are:
        \3 DAMIP e.g. \autocite{querec2009SystemsBiologyApproach}, which gives rulesets composed of small sets of genes
        \3 elastic net
        % https://www.ncbi.nlm.nih.gov/pmc/articles/PMC2810828/
        % Sparse partial least squares regression for simultaneous dimension reduction and variable selection
        % https://pubmed.ncbi.nlm.nih.gov/20361856/
        % Sparse Partial Least Squares Classification for High Dimensional Data
        \3 sparse partial least squares e.g. \autocite{tsang2014GlobalAnalysesHuman} 
    \2 Note the sparsity assumption for these methods (most genes are not predictive) is in opposition to the idea of combining many weak predictors to maximise predictive power, but will be less costly to translate to clinical practice

% hard to imagine Personalising
% e.g. Vaccines
    % but a decade ago, only TIV.

\section{From association to causality}

% TODO are these reQTL interesting?
% In vivo reqtl Design can find genes and contexts that are somewhat relevant to post vacc
% But does it find ones that are causally important???

% gwas often not possible
% direct signal would have helped unlock more traditional pipeline start starts anchored in causality

\1 uncovering \enquote{mechanism}
    \2 Knowing the molecular mechanisms of response are crucial for conceiving of possible interventions.
        \3 e.g. ch4 baseline SIGLEC10 expression is associated with anti-TNF response, but would changing baseline SIGLEC10 affect response?
    \2 In observational data, a causally upstream anchor is required: genetic variation
    \2 the regression analyses in this thesis:
        \3 DGE describes expression response association with phenotypic response (such as Ab titre) 
        \3 reQTLs describe genotype association with expression response
    % \2 rather than simple overlap,
    % \2 but the point is to learn if there are genetic effects on response trait VIA expression response, otherwise you'd just do the GWAS
    % \2 guarded against two types of reverse causal fx, but still need formal integration
    \2 Need a model that encodes causal assumptions integrating multiple data layers: genotype -> gene expression -> phenotypic response
    \2 there are several families of complementary methods that integrate multiple data layers

% TODO: read this intro https://www.biorxiv.org/content/10.1101/2020.07.01.182097v1.full.pdf
%
% TODO: be specific on what sort of pleiotropy MR can distinguish
%
% TODO:
% https://www.nature.com/articles/s41588-020-0682-6?s=09
% Predicted associations between proteins and phenotypes may indicate four explanations: causality, reverse causality, confounding by LD between the leading SNPs for proteins and phenotypes or horizontal pleiotropy (Supplementary Fig. 3). Given these alternative explanations, we conducted a set of sensitivity analyses to evaluate whether each MR association reflected a causal effect of protein on phenotype: tests of reverse causality using bidirectional MR22 and MR Steiger filtering;23,24 heterogeneity analyses for proteins with multiple instruments25; and colocalization analyses26 to investigate whether the genetic associations with both protein and phenotype shared the same causal variant (Fig. 1).
%
% TODO:
% https://www.sciencedirect.com/science/article/pii/S0168952520302092?s=09
% In the past few years different methods to jointly evaluate eQTL and GWAS data have been developed. One class of approaches, broadly termed colocalization, focuses on the associated loci themselves. Colocalization approaches test the hypothesis that a shared variant causally impacts on both the disease and gene expression [12., 13., 14., 15.], providing stronger evidence that the regulatory effect underlies the disease mechanism. However, colocalization alone cannot distinguish between a variant that influences expression and disease separately (‘pleiotropy’) and one that affects disease directly via a regulatory effect (‘mediation’) [16]. A complementary set of methods focuses on the associated genes instead of the specific loci. These approaches test the association between genetically influenced gene expression levels and disease, using eQTL data to construct a predictor of gene expression that can be evaluated for association with disease in GWAS studies. In essence, reference eQTL data, that relate genotypes to expression, are employed to interpret GWAS data that relate genotypes to disease but have no direct measurements of gene expression. Notable examples include PrediXcan and TWAS, which compare predicted expression with disease status to prioritize disease genes, and Mendelian randomization, which uses eQTL data as an instrumental variable for two-step least-squares regression to evaluate the effect of gene expression on disease [17., 18., 19., 20.].
%
%
% Other types of methods:
% Summary: https://sashagusev.github.io/2017-10/twas-vulnerabilities.html
% transcriptome association (e.g. TWAS)
%    https://www.ncbi.nlm.nih.gov/pmc/articles/PMC6342197/
%    As TWAS methods were originally proposed as tests for association between
%    local genetically regulated component of expression and disease with no
%    causality guarantees [Gamazon, et al. 2015; Gusev, et al. 2016; Mancuso,
%    et al. 2017a; Mancuso, et al. 2017b; Zhu, et al. 2016], it remains unclear
%    whether and when TWAS can be interpreted as valid tests of causality.
%
%     https://www.nature.com/articles/s41576-018-0020-3
%     Transcriptome-wide association studies (TWAS) integrating GWAS and eQTLs
%     data have been proposed to unravel gene–trait associations7,9,10. However,
%     although these studies aim to identify genes whose (genetically predicted)
%     expression is significantly associated to complex traits, they do not aim
%     to estimate the strength of the causal effect and are unable to distinguish
%     causation from horizontal pleiotropy (i.e., when a genetic variant
%     influences multiple phenotypes independently).
%
%     https://www.ncbi.nlm.nih.gov/pmc/articles/PMC5986723/
%     Like other transcriptome-wide association studies (TWASs),32 PrediXcan can be
%     considered a weighted burden test, where each variant in a gene set is weighted
%     by its additive allelic effect on expression.
%
% PPV of TWAS is low for assigning causal genes
% https://pubmed.ncbi.nlm.nih.gov/31978332/
%
% EDS Framework Complicates eQTL Prioritization of Genes at GWAS Loci
% https://www.cell.com/ajhg/fulltext/S0002-9297(20)30012-4
%
% Mendelian randomisation (e.g. MR-Egger, SMR) under certain assumptions
%     hemani2018EvaluatingPotentialRole
%     Of prime focus among the many limitations to MR is the unprovable assumption
%     that apparent pleiotropic associations are mediated by the exposure (i.e.
%     reflect vertical pleiotropy), and do not arise due to SNPs influencing the two
%     traits through independent pathways (‘horizontal pleiotropy’)
%
% Mediation
%     hemani2018EvaluatingPotentialRole
%     Genetic mediation-based analyses (37–40) are more liable to problems of
%     confounding and measurement error than MR (41–43), but could potentially
%     separate between vertical and horizontal pleiotropy in some scenarios.
%     Use genetic colocalization to eliminate possibility distinct causal variants
%     (25,30,31); if instruments are available for the outcome then test the reverse
%     causal effect (110); if not use MR Steiger (43); use genetic mediation-based
%     analysis (40,111) to try to separate horizontal and vertical pleiotropy
%
%     millstein2009DisentanglingMolecularRelationships
%     Causal effect estimates often
%     considered in 'Mendelian randomization' approaches [11], can be confounded by
%     pleiotropic effects and reverse causation [12], thus, these approaches are not
%     generally considered for problems such as reconstructing transcript regulatory
%     pathways, in which pleiotropy is common and there may be little a priori
%     information on the structure of the causal relationship between traits.

\1 <Mendelian randomisation>
    \2 <explain the analogy to RCTs: segregation should be independent of env factors>
    \2 is a type of IV analysis that uses genetic instruments: can use eQTLs as instrumental variables (IV) to estimate causal effect between expression (exposure) and phenotype (outcome)
    \2 three assumptions that define a valid IV \autocite{daveysmith2014MendelianRandomizationGenetic,hemani2018EvaluatingPotentialRole,neumeyer2020StrengtheningCausalInference}
        \3 IV1 SNP is associated with expression i.e. an eQTL (no need to be fine mapped \autocite{daveysmith2014MendelianRandomizationGenetic,burgess2018InferringCausalRelationships})
        \3 IV2 SNP is not associated with confounders of the expression-phenotype association 
        \3 IV3 SNP has no association with phenotype except through expression 
    \2 combined, these assumptions mean that expression is a complete mediator of the causal effect of the SNP on phenotype (a vertical pleiotropy is assumed)
    \2 two-sample MR is were G -> E and E -> pheno are estimated in two non-overlapping samples \autocite{hemani2018EvaluatingPotentialRole,neumeyer2020StrengtheningCausalInference}, and can come from large existing QTL datasets
    % \2 Also avoids weak-instrument bias, where G -> E is estimated in a small sample
        \3 a related family of methods, TWAS \url{https://www.nature.com/articles/ng.3506}, tests association of genetically predicted expression to phenotypes, have their methodological basis in two-sample MR \url{https://link.springer.com/article/10.1007/s40484-020-0207-4}

\1 assumption violations will lead to bias in estimating causal effect of E on pheno
    \2 IV1 is satisfied by definition of an eQTL
    \2 IV2 violations are not likely, as genetic variants are unlikely to be associated with environmental confounders \todo{citation needed, try daveysmith2020MendelLawsMendelian}
    \2 IV3 violation is most likely

% coloc distinguishes pleiotropy from linkage, but not vertical pleiotropy (mediation) from horizontal pleiotropy (independent effects on trait and expression).
% As colocalisation of a \gls{GWAS} loci with \glspl{eQTL} is is necessary but not sufficient for mediation,
% it should be supported by complementary lines of evidence to help untangle the multiplex possible causal pathways from variant to trait \autocite{hemani2018EvaluatingPotentialRole}.
\1 <coloc>
    \2 IV3 violated by linkage: if the instrument eQTL is actually not causal, but simply correlated with another variant that causally affects phenotype
        \3 use colocalisation methods to test whether the same variants are associated with E and phenotype \autocite{hemani2018EvaluatingPotentialRole}
        \3 described in ch1, and used in ch3

\1 <mediation analysis>
    \2 IV3 violated by horizontal pleiotropy: instrument eQTL causes expression and phenotype independently (e.g. by also being an eQTL for another gene whose expression causally affects pheno), such that expression and phenotype are not causally related
    \2 use mediation analysis e.g. (CIT\autocite{millstein2009DisentanglingMolecularRelationships}, findr\autocite{wang2017EfficientAccurateCausal}) that model comparison to distinguish between models of horizontal from vertical pleiotropy
        \3 \autocite{millstein2009DisentanglingMolecularRelationships} used by \textcite{franco2013IntegrativeGenomicAnalysis} in a vaccine response reQTL setting, although they concluded they were underpowered
    \2 however, requires individual level data, and more susceptible to measurement error than MR \autocite{hemani2018EvaluatingPotentialRole}

\1 IV3 violated by reverse causation: assumed expression -> response, but the causal direction might actually be the reverse: G -> phenotype -> expression
    \2 if suspected, recommendations to distinguish \autocite{daveysmith2014MendelianRandomizationGenetic,hemani2018EvaluatingPotentialRole,neumeyer2020StrengtheningCausalInference} \url{https://www.ncbi.nlm.nih.gov/pmc/articles/PMC5036871/}
        \3 perform MR in the reverse direction if there are instruments for the phenotype (bi-directional MR)
        \3 extensions that test the directionality based on variance explained (MR Steiger) \url{https://journals.plos.org/plosgenetics/article?id=10.1371/journal.pgen.1007081}
        \3 incorporate biological knowledge, e.g. if eQTL variant is in a TF motif for a gene, more likely G -> E than G -> pheno -> E

% NOTE: https://www.ncbi.nlm.nih.gov/pmc/articles/PMC7221461/
% has the nicest MR diagrams to copy.

% Mediation can exist in the absence of a 'significant' total effect G -> pheno
% http://davidakenny.net/cm/mediate.htm
% In the opinion of most though not all analysts, Step 1 is not required. (See the Power section below why the test of c can be low power, even if paths a and b are non-trivial.)

% Canalisation/compensation and population stratification affects MR with genetic instruments also.
% https://www.nature.com/articles/s41576-018-0020-3
% https://spiral.imperial.ac.uk/bitstream/10044/1/23560/2/SIM%202015_Accepted%20version.pdf
    % The most likely cause of violation of the third IV assumption is pleiotropy,  which occurswhen a genetic variant influences multiple intermediate phenotypes that separately affectthe outcome of interest.
    % The assumption can be violated through other mechanisms includ-ing canalization and population stratification.
    % Developmental canalization occurs when agenetic variant expressed during fetal development or post-natal growth stimulates compen-satory processes that protect against the effect of that variant on the outcome in adulthood[1], while population stratification occurs when ancestral sub-populations with different al-lele frequency and outcome distributions confound theG−Yassociation [1, 5, 6].

% Leland
% \autocite{taylor2019IntegrativeAnalysisGene}
% MR is a statistical framework that uses a genetic association (“the
    % instrument”) for one trait (“the exposure”) to test for a causal
    % influence on another trait (“the outcome”).  An MR result, in which the
    % exposure instrument is associated with the outcome, can arise under four
    % distinct models (10, 11): (i) there is no causal relationship, but a SNV
    % that influences the outcome is in linkage disequilibrium (LD) with a SNV
    % that influences the exposure; (ii) the exposure causally influences the
    % outcome; (iii) the outcome causally influences the exposure (a reverse
    % causal relationship); or (iv) the exposure and outcome are not causally
    % related but share a SNV that influences both the exposure and the outcome
    % independently (horizontal pleiotropy).

\section{Triangulation}

\1 Triangulation is the use of methods with different assumptions, biases, and limitations that address the same question \autocite{munafo2018RobustResearchNeeds}
    \2 discussed above is one example of triangulation, using complementary methods MR, colocalisation, mediation analysis to tease apart causal mechanisms from G -> E -> pheno
        % To distinguish among these models, we use four complementary tests (defined
        % here; Fig. S18): an MR test (consistent with all models), a
        % colocalization test (distinguishing model i from models ii–iv; Note S1),
        % the MR Steiger test (distinguishing model ii from model iii), and the
        % causal inference test (CIT; distinguishing among models ii–iv).
        \3 e.g. \textcite{taylor2019IntegrativeAnalysisGene}: an example combining MR, coloc (HEIDI), MR Steiger, CIT, to disentangle causal effect of SNPs on DNA Me and expression
        % TODO specific distinguishes
        \3 \autocite{zheng2019PhenomewideMendelianRandomization} Predicted associations between proteins and phenotypes may indicate four explanations: causality, reverse causality, confounding by LD between the leading SNPs for proteins and phenotypes or horizontal pleiotropy (Supplementary Fig. 3). Given these alternative explanations, we conducted a set of sensitivity analyses to evaluate whether each MR association reflected a causal effect of protein on phenotype: tests of reverse causality using bidirectional MR22 and MR Steiger filtering;23,24 heterogeneity analyses for proteins with multiple instruments25; and colocalization analyses26 to investigate whether the genetic associations with both protein and phenotype shared the same causal variant (Fig. 1). 
    \2 another example, how G -> E effect changes in response to perturbation
        \3 combine reQTL mapping (between individuals), dynamic ASE (within heterozygous individuals), (and maybe variance QTLs)
        % Allele-specific expression changes dynamically during T cell activation in HLA and other autoimmune loci
        \3 dynamic ASE example for multiple T cell activation states \url{https://www.nature.com/articles/s41588-020-0579-4}
    \1 another example, multiomics integration
        \2 different layers have non-redundant readouts
        \2 e.g. pQTLs
        % https://www.nature.com/articles/s41576-020-0258-4
        % Although protein and mRNA levels typically show reasonable correlation, we describe how transcriptomics and proteomics provide useful non-redundant readouts.
        \2 e.g. foreshadowing of eQTL by caQTLs 
        % https://www.nature.com/articles/s41588-018-0046-7
        \2 network approaches, MOFA
\1 many limitations discussed cannot be solved by increasing n
\1 triangulation will be key in moving from a descriptive to mechanistic understanding of immune response to perturbation

FINAL CCC
This thesis only scratches the surface
WHY DID IT IN FIRST PLACE
...
HOW CONTRBUTE TO FIELD
% TODO
%        \3 future outlook for the fields of systems studies of immune response% vaccinogenomics and pharmacogenomics
% there is benefit in the GWAS approach meeting the sysvacc approach once sample sizes are enough

\end{outline}

old adage
principles of immunology,  perturb observe and understand
    larger samples
    rich sapmling designs
    better analysis
    new tech
long road from perturbation to understanding indeed
will be 

%
% Misc
%
% Network analysis
    % https://bmcgenomics.biomedcentral.com/articles/10.1186/1471-2164-13-356
    % Beyond differential expression: the quest for causal mutations and effector molecules
    % In trying to understand why a phenotype changes, one should not merely think “which gene(s) are the most differentially expressed”, but rather “which gene(s) are the most differentially connected.”
    % This insight introduces us to the field of network science.
%
% Large resources 
    % e.g. dejager2015ImmVarProjectInsights, zalocusky201810000Immunomes
%
% Longitudinal
%     Finally, note that a longitudinal design can provide measurements of gene expression at many timepoints
%     Causal inference with a time-varying exposure is even harder.

