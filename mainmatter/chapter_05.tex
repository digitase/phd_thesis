%
% Chapter 5
%

\chapter{multiPANTS}

\section{Introduction}

Why do some people not respond?

Explore time series transcriptomic

Multilevel model where individual is a RE, 
Find out optimal spline degree.
Then work out if genetics changes trajectories for any gene i.e. DGE models with a snp as predictor
First need to eQTL scan in general with mashr and find the snps in the most reQTLish genes, since this modelling is probably expensive

Creating composite features to conduct genetic associations on.

Identifying signatures of response.

\section{Methods}

% https://discourse.datamethods.org/t/responder-analysis-loser-x-4/1262
Responder analysis is a 4-fold loser. It fails to give the needed clinical
interpretation, has poor statistical properties, increases the cost of studies
because binary outcomes require significantly higher sample sizes, and raises
ethical issues because more patients are required to be randomized than would
be needed were the endpoint and analysis to be fully efficient.

immunomods
% > pheno %>% filter(RNAseq_Sample_Status == 'Pass') %>% distinct(PANTS.ID, .keep_all=T) %>% select(Primary_Response, Ever_Immunomodulator) %>% table
%                 Ever_Immunomodulator
% Primary_Response  NO YES
%            FALSE  24 129
%            TRUE   28 159
% > pheno %>% filter(RNAseq_Sample_Status == 'Pass') %>% distinct(PANTS.ID, .keep_all=T) %>% select(Primary_Response, On_Immunomodulator_At_Baseline) %>% table
%                 On_Immunomodulator_At_Baseline
% Primary_Response FALSE TRUE
%            FALSE    85   68
%            TRUE     83  104
% > pheno %>% filter(RNAseq_Sample_Status == 'Pass') %>% distinct(PANTS.ID, .keep_all=T) %>% select(Primary_Response, On_Steroids_At_Baseline) %>% table
%                 On_Steroids_At_Baseline
% Primary_Response FALSE TRUE
%            FALSE    83   70
%            TRUE    133   54

In the IFX+ADA cohort, 
    DE
    PR vs PNR baseline
    PR vs PNR and w14

n patients with data for each number of visits

\subsection{Covariates to use}

Sex
Age
BMI
Age of Onset
Crohn’s Surgery
Ever Immunomodulator
Current Smoker
PCA
Proportions of the 6 cell types: CD4+ T cells, CD8+ T cells, B cells, NK cells, monocytes, and granulocytes

\subsection{reQTL}

% Alternative:
% In the ANCOVA approach, the whole focus is on whether one group has a higher mean after the treatment. It’s appropriate when the research question is not about gains, growth, or changes.

% https://www.theanalysisfactor.com/pre-post-data-repeated-measures/
ANCOVA vs repeated measures vs mixed model

% The biggest advantage of mixed models is their incredible flexibility.  They can handle clustered individuals as well as repeated measures (even in the same model).  They can handle crossed random effects, where there are repeated measures not only on an individual, but also on each stimulus.
% https://www.theanalysisfactor.com/repeated-measures-approaches/

\section{Results}

\section{Discussion}

