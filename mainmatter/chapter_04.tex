%
% Chapter 4
%

\chapter{Response to live attenuated rotavirus vaccine (Rotarix) in Vietnamese infants}

\section{Introduction}

Summary 

Rotavirus vaccine efficacy is lower in LMICs than EU and NA.
Protective response to many vaccines is linked with genetic variation.
Hypothesis: difference in efficacy is due to differences in genetic variation.

Aim:
    identify genetic and transcriptomic markers associated with Rotarix protective response
    primary outcome will be Rotarix vaccine failure events 
    secondary outcomes will be antibody responses and genotypic characterization of the infection virus in Rotarix failure events

\subsection{The genetics of vaccine response in early life}

\subsection{Rotavirus and rotarix in Vietnam}

\subsection{Known factors that affect rotavirus vaccine efficacy}

\section{Methods}

\subsection{RNA-seq data generation}

Stranded RNAseq AUTO with Globin Depletion (>47 samples) uses the NEB Ultra
II directional RNA library kit for the poly(A) pulldown, fragmentation, 1st and
2nd strand synthesis and the flowing cDNA library prep (with some minor tweaks
e.g. at during the PCR we use kapa HiFi not NEB's Q5 polymerase). Between the
poly (A) pulldown and the fragmentation we use a kapa globin depletion kit
(it's very similar to their riboerase kit but the rRNA probes are swapped for
globin ones).

\subsection{Genotyping}

We will also use the SNP data to accurately impute ABO blood groups and secretor status. 

\section{Results}

Transcriptomic response to rotavirus vaccination (pre- vs. post-, prime vs. boost dose, responders vs. non-responders)

Genetic contribution to transcriptomic response

\section{Discussion}

