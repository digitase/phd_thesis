%
% Abstract
%

\chapter{Abstract}

% \Large{Genomic profiling of response to \textit{in vivo} immune perturbations}

% \large{Benjamin Yu Hang Bai}

The human immune system plays a central role in defense against infection, 
but its dysregulation is implicated in immune-mediated diseases.
The past decade has seen increasing application of high-throughput technologies to profile, predict, and understand immune response to perturbation.
% This thesis examines response to two immune perturbations:
% influenza vaccine response in healthy individuals,
% and drug response in patients with immune-mediated disease.
The ability to measure immune gene expression at scale has led to the identification of 
transcriptomic signatures that predict clinical phenotypes such as antibody response to vaccines.
It has also been recognised that both expression and phenotypic responses are traits with complex genetic architectures.
This thesis examines the longitudinal transcriptomic response to immune perturbations,
and its association with clinical response phenotypes and common genetic variation.

\Cref{ch:hird_DGE} explores transcriptomic response to pandemic influenza vaccine in a multi-ethnic cohort of healthy adults: the \gls{HIRD} cohort.
The success of vaccination in controlling influenza is indisputable, 
but it is incompletely understood why some individuals fail to mount protective antibody responses.
I meta-analysed blood microarray and \gls{RNAseq} datasets, 
identifying a distinct transition from innate immune response at day 1 after vaccination to adaptive immune response at day 7.
Heterogeneity between measurement platforms made it difficult to identify single-gene transcriptomic associations with antibody response.
Using a gene set approach, I found expression modules related to the inflammatory response, the cell cycle, CD4\textsuperscript{+} T cells, and plasma cells 
to be associated with vaccine-induced antibody response.

In \cref{ch:hird_reQTL}, I map \glspl{reQTL} in the \gls{HIRD} cohort to investigate regulation of transcriptomic response by common genetic variants.
Rather than driving differential expression post-vaccination,
the strongest \glspl{reQTL} appeared to be explained by changes in cell composition revealing cell type-specific \gls{eQTL} effects.
For example, a \gls{reQTL} identified for \gene{ADCY3} specific to day 1 may be explained largely by high monocyte proportions at day 1 compared to other timepoints.
Changes in cell composition present a significant challenge to interpreting \glspl{reQTL} found through bulk sequencing of heterogeneous tissues.

Finally, \cref{ch:multiPANTS} applies an analogous longitudinal study design to explore drug response in the \gls{PANTS} cohort,
a cohort of \gls{CD} patients treated with the anti-\gls{TNF} drugs, infliximab and adalimumab.
Anti-\gls{TNF} treatment has revolutionised patient care for \gls{CD},
but \SIrange{20}{40}{\percent} of patients show primary non-response soon after starting treatment.
I identified baseline expression modules associated with primary non-response, but also found significant heterogeneity of associations between the two drugs.
Expression changes post-treatment in non-responders were largely magnified in responders, suggesting there may be a continuum of response.
Distinct expression trajectories identified for responders and non-responders revealed sustained expression differences during the first year of treatment.
A set of interferon-related genes were regulated in opposing directions in responders and non-responders,
presenting an attractive target for future studies of the biological mechanisms underlying non-response.
%
