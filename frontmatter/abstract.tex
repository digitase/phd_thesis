%
% Abstract
%
\chapter*{Abstract}

\begin{outline}

\1 <working title>
    \2 High-throughput genomic profiling of response to \textit{in vivo} immune perturbations
\1 <other candidate titles>
    \2 Transcriptomic response to immune perturbation, and its genetic architecture
    \2 Characterising immune response to vaccines and drugs using longitudinal \textit{in vivo} study designs
    \2 Identifying associations with vaccine and drug response in longitudinal studies
    \2 Systems immunology of \textit{in vivo} vaccine and drug responses

\end{outline}

The human immune system plays a central role in defense against infection, 
but it's dysregulation is implicated in immune-mediated diseases.
The past decade has seen increasing application of high-throughput technologies to profile, predict, and understand immune response to perturbation.
% This thesis examines response to two immune perturbations:
% influenza vaccine response in healthy individuals,
% and drug response in patients with immune-mediated disease.
The ability to measure immune gene expression at scale has led to the identification of 
transcriptomic signatures that predict clinical phenotypes such antibody response to vaccines.
It has also been recognised that both expression and phenotypic responses are complex traits with their own genetic architecture.
This thesis examines the longitudinal transcriptomic response to immune perturbations,
and it's association with clinical response phenotypes and common genetic variation.

\Cref{ch:hird_DGE} explores transcriptomic response to pandemic influenza vaccine in a multi-ethnic cohort of healthy adults.
The success of vaccination in controlling influenza is indisputable, 
but it is incompletely understood why some individuals fail to mount protective antibody responses.
I meta-analysed blood microarray and \gls{RNAseq} datasets, 
identifying a distinct transition from innate immune response at day 1 after vaccination, to adaptive immune response at day 7.
Heterogeneity between measurement platforms made it difficult to identify single-gene associations with antibody response.
Using a gene set approach, I found expression modules related to inflammatory response, the cell cycle, CD4+ T cells, and plasma cells 
to be associated with vaccine-induced antibody response.

In \cref{ch:hird_reQTL}, I map \glspl{reQTL} in the same cohort to investigate regulation of transcriptomic response by common genetic variants.
Rather than driving differential expression post-vaccination,
the strongest \gls{reQTL} appear to be explained by changes in cell composition revealing cell-type specific \glspl{eQTL}.
For example, a \gls{reQTL} identified for \gene{ADCY3} specific to day 1 was driven largely by high monocyte proportions at day 1.
Changes in cell composition present a significant challenge to the interpretation of \glspl{reQTL} found in bulk data.

Finally, \cref{ch:multiPANTS} applies an analogous longitudinal study design to explore \gls{CD} patient response to anti-\gls{TNF} drugs: infliximab and adalimumab.
Anti-\gls{TNF} treatment has revolutionised patient care for \gls{CD},
but 20-40\% of patients develop primary non-response soon after starting treatment.
I identified baseline expression modules associated with primary response, but also significant heterogeneity in these associations between drugs.
Expression changes post-treatment in non-responders were largely magnified in responders, suggesting there may be a continuum of response.
Distinct expression trajectories identified for responders and non-responders revealed sustained expression differences up to week 54.
A set of interferon-related genes were regulated in opposing directions in responders and non-responders,
presenting an attractive target for future studies of the biological mechanisms of non-response.

