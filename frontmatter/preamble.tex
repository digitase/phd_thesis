%
% Preamble
%
% https://tex.stackexchange.com/questions/553/what-packages-do-people-load-by-default-in-latex

%
% Typesetting
%

% https://tex.stackexchange.com/questions/44694/fontenc-vs-inputenc
%
% Use modern font encoding
% fontenc is oriented to output, that is, what fonts to use for printing characters.
\usepackage[T1]{fontenc}
% inputenc allows the user to input accented characters directly from the keyboard
\usepackage[utf8]{inputenc}

% Correct hyphenation
% Specify patterns using e.g.  \hyphenation{hy-phen-a-tion MATLAB}
\usepackage[english]{babel}

% Typographic optimisation
% It doesn't work with latex, you have to use pdflatex instead.
\usepackage{microtype}

% Better Computer Modern font
\usepackage{lmodern}

% Extension package to amsmath
\usepackage{mathtools}
% Extended symbol collection
\usepackage{amssymb}
% For theorems
\usepackage{amsthm}

%
% Figures
%

\usepackage{graphicx}
\usepackage[font={footnotesize}]{caption}
\usepackage[font={footnotesize}]{subcaption} %Allow subfigures
% Changes the filename algorithm to check for known graphics extensions
\usepackage{grffile} 

% Nicer tables
\usepackage{booktabs}

% Pretty-print code
\usepackage{listings}

% Allow modifications to enumerate lists
\usepackage{enumitem}

%
% Formatting
%

% Format numbers using e.g. \numprint{2.742647826672E-01}
\usepackage{numprint}

% Format si units
\usepackage{siunitx}

% Define line spacing
% Setspace switches off double-spacing at places where even the most die-hard
% official would doubt its utility (footnotes, figure captions, and so on); it’s
% very difficult to do this consistently if you’re manipulating \baselinestretch
% yourself.
\usepackage{setspace}
\onehalfspacing

%
% Navigation
%

% Automatically generate hypertext links between
% the acronyms in the text and their definition in the list of acronyms
\usepackage{acronym}

% Clickable links
\usepackage{hyperref}

% TO-DO command and list
\usepackage{todonotes}

%
% Bibliography
%
% https://tex.stackexchange.com/questions/25701/bibtex-vs-biber-and-biblatex-vs-natbib
%
% bibtex and biber are external programs that process bibliography information
% and act (roughly) as the interface between your .bib file and your LaTeX
% document.
% natbib and biblatex are LaTeX packages that format citations and
% bibliographies; natbib works only with bibtex, while biblatex (at the
% moment) works with both bibtex and biber.)

% https://leesjohn.wordpress.com/2013/04/22/a-latex-bibliography-style-i-like-nature-style-in-biblatex/
\usepackage[
	style=nature,
	maxcitenames=1,
	maxnames=6,
	firstinits=true, 
	uniquename=init,
	sorting=none,
	url=false,
	doi=true,
	isbn=false,
	eprint=false,
	texencoding=utf8,
	bibencoding=utf8,
	autocite=superscript,
	backend=biber
]{biblatex}

% When using babel or polyglossia with biblatex, loading csquotes is recommended
% to ensure that quoted texts are typeset according to the rules of your main
% language.
\usepackage{csquotes}

\addbibresource{./backmatter/zotero_My-Library_Better-BibLaTeX.bib}
